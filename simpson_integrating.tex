\h1{Интегрирование по формуле Симпсона}

Требуется посчитать значение определённого интеграла:

$$ \int_a^b f(x) dx $$

Решение, описываемое здесь, было опубликовано в одной из диссертаций \bf{Томаса Симпсона} (Thomas Simpson) в 1743 г.


\h2{Формула Симпсона}

Пусть $n$ --- некоторое натуральное число. Разобьём отрезок интегрирования $[a;b]$ на $2n$ равных частей:

$$ x_i = a + i h,~~i = 0 \ldots 2n, $$
$$ h = \frac{ b-a }{ 2n }. $$

Теперь посчитаем интеграл отдельно на каждом из отрезков $[x_{2i-2}, x_{2i}], i = 1 \ldots n$, а затем сложим все значения.

Итак, пусть мы рассматриваем очередной отрезок $[x_{2i-2}, x_{2i}], i = 1 \ldots n$. Заменим функцию $f(x)$ на нём параболой, проходящей через 3 точки $(x_{2i-2},x_{2i-1},x_{2i})$. Такая парабола всегда существует и единственна. Её можно найти аналитически, затем останется только проинтегрировать выражение для неё, и окончательно получаем:

$$ \int_{x_{2i-2}}^{x_{2i}} f(x) dx = \left( f(x_{2i-2}) + 4 f(x_{2i-1}) + f(x_{2i}) \right) \frac{ h }{ 3 }. $$

Складывая эти значения по всем отрезкам, получаем окончательную \bf{формулу Симпсона}:

$$ \int_a^b f(x) dx = \left( f(x_0) + 4 f(x_1) + 2 f(x_2) + 4 f(x_3) + 2 f(x_4) + \ldots + 4 f(x_{2N-1}) + f(x_{2N}) \right) \frac{ h }{ 3 }. $$


\h2{Погрешность}

Погрешность, даваемая формулой Симпсона, не превосходит по модулю величины:

$$ \frac{ 1 }{ 180 } h^4 (b - a) \max_{a \le x \le b} \left| f^{(4)}(x) \right|. $$

Таким образом, погрешность имеет порядок уменьшения как $O (n^4)$.


\h2{Реализация}

Здесь $f(x)$ --- некоторая пользовательская функция.

\code
double a, b; // входные данные
const int N = 1000*1000; // количество шагов (уже умноженное на 2)
double s = 0;
double h = (b - a) / N;
for (int i=0; i<=N; ++i) {
	double x = a + h * i;
	s += f(x) * ((i==0 || i==N) ? 1 : ((i&1)==0) ? 2 : 4);
}
s *= h / 3;
\endcode