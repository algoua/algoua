\h1{Оптимальный выбор заданий при известных временах завершения и длительностях выполнения}

Пусть дан набор заданий, у каждого задания известен момент времени, к которому это задание нужно завершить, и длительность выполнения этого задания. Процесс выполнения какого-либо задания нельзя прерывать до его завершения. Требуется составить такое расписание, чтобы выполнить наибольшее число заданий.

\h2{Решение}

Алгоритм решения --- \bf{жадный} (greedy). Отсортируем все задания по их крайнему сроку, и будем рассматривать их по очереди в порядке убывания крайнего срока. Также создадим очередь $q$, в которую мы будем постепенно помещать задания, и извлекать из очереди задание с наименьшим временем выполнения (например, можно использовать set или priority_queue). Изначально $q$ пустая.

Пусть мы рассматриваем $i$-ое задание. Сначала поместим его в $q$. Рассмотрим отрезок времени между сроком завершения $i$-го задания и сроком завершения $i-1$-го задания --- это отрезок некоторой длины $T$. Будем извлекать из $q$ задания (в порядке увеличения оставшегося времени их выполнения) и помещать на выполнение в этом отрезке, пока не заполним весь отрезок $T$. Важный момент --- если в какой-то момент времени очередное извлечённое из структуры задание можно успеть частично выполнить в отрезке $T$, то мы выполняем это задание частично --- именно настолько, насколько это возможно, т.е. в течение $T$ единиц времени, а оставшуюся часть задания помещаем обратно в $q$.

По окончании этого алгоритма мы выберем оптимальное решение (или, по крайней мере, одно из нескольких решений). Асимптотика решения --- $O (n \log n)$.


\h2{Реализация}


\code
int n;
vector < pair<int,int> > a; // задания в виде пар (крайний срок, длительность)
... чтение n и a ...

sort (a.begin(), a.end());

typedef set < pair<int,int> > t_s;
t_s s;
vector<int> result;
for (int i=n-1; i>=0; --i) {
	int t = a[i].first - (i ? a[i-1].first : 0);
	s.insert (make_pair (a[i].second, i));
	while (t && !s.empty()) {
		t_s::iterator it = s.begin();
		if (it->first <= t) {
			t -= it->first;
			result.push_back (it->second);
		}
		else {
			s.insert (make_pair (it->first - t, it->second));
			t = 0;
		}
		s.erase (it);
	}
}

for (size_t i=0; i<result.size(); ++i)
	cout << result[i] << ' ';
\endcode