<h1>Матричная теорема Кирхгофа. Нахождение количества остовных деревьев</h1>
<p>Задан связный неориентированный граф своей матрицей смежности. Кратные рёбра в графе допускаются. Требуется посчитать количество различных остовных деревьев этого графа.</p>
<p>Приведённая ниже формула принадлежит Кирхгофу (Kirchhoff), который доказал её в 1847 г.</p>
<h2>Матричная теорема Кирхгофа</h2>
<p>Возьмём матрицу смежности графа G, заменим каждый элемент этой матрицы на противоположный, а на диагонале вместо элемента A<sub>i,i</sub> поставим степень вершины i (если имеются кратные рёбра, то в степени вершины они учитываются со своей кратностью). Тогда, согласно матричной теореме Кирхгофа, все алгебраические дополнения этой матрицы равны между собой, и равны количеству остовных деревьев этого графа. Например, можно удалить последнюю строку и последний столбец этой матрицы, и модуль её определителя будет равен искомому количеству.</p>
<p>Определитель матрицы можно найти за O (N<sup>3</sup>) с помощью <algohref=determinant_gauss>метода Гаусса</algohref> или <algohref=determinant_crout>метода Краута</algohref>.</p>
<p>Доказательство этой теоремы достаточно сложно и здесь не приводится (см., например, Приезжев В.Б. "Задача о димерах и теорема Кирхгофа").</p>
<h2>Связь с законами Кирхгофа в электрической цепи</h2>
<p>Между матричной теоремой Кирхгофа и законами Кирхгофа для электрической цепи имеется удивительная связь.</p>
<p>Можно показать (как следствие из закона Ома и первого закона Кирхгофа), что сопротивление R<sub>ij</sub> между точками i и j электрической цепи равно:</p>
<formula>R<sub>ij</sub> = |T<sup>(i,j)</sup>| / |T<sup>j</sup>|</formula>
<p>где матрица T получена из матрицы A <i>обратных</i> сопротивлений проводников (A<sub>ij</sub> - обратное число к сопротивлению проводника между точками i и j) преобразованием, описанным в матричной теореме Кирхгофа, а обозначение T<sup>(i)</sup> обозначает вычёркивание строки и столбца с номером i, а T<sup>(i,j)</sup> - вычёркивание двух строк и столбцов i и j.</p>
<p>Теорема Кирхгофа придаёт этой формуле геометрический смысл.</p>