\h1{ Вершинная связность. Свойства и нахождение }


\h2{ Определение }

Пусть дан неориентированный граф $G$ с $n$ вершинами и $m$ рёбрами.

\bf{Вершинной связностью} $\lambda$ графа $G$ называется наименьшее число вершин, которое нужно удалить, чтобы граф перестал быть связным.

Например, для несвязного графа вершинная связность равна нулю. Для связного графа с единственной точкой сочленения вершинная связность равна единице. Для полного графа вершинную связность полагают равной $n-1$ (поскольку, какую пару вершин мы ни выберем, даже удаление всех остальных вершин не сделает их несвязными). Для всех графов, кроме полного, вершинная связность не превосходит $n-2$ --- поскольку можно найти пару вершин, между которыми нет ребра, и удалить все остальные $n-2$ вершины.

Говорят, что множество $S$ вершин \bf{разделяет} вершины $s$ и $t$, если при удалении этих вершин из графа вершины $u$ и $v$ оказываются в разных компонентах связности.

Ясно, что вершинная связность графа равна минимуму от наименьшего числа вершин, разделяющих две вершины $s$ и $t$, взятому среди всевозможных пар $(s,t)$.


\h2{ Свойства }


\h3{ Соотношение Уитни }

\bf{Соотношение Уитни (Whitney)} (1932 г.) между \algohref=rib_connectivity{рёберной связностью} $\lambda$, вершинной связностью $\kappa$ и наименьшей из степеней вершин $\delta$:

$$ \kappa \le \lambda \le \delta. $$

\bf{Докажем} это утверждение.

Докажем сначала первое неравенство: $\kappa \le \lambda$. Рассмотрим этот набор из $\lambda$ рёбер, делающих граф несвязным. Если мы возьмём от каждого из этих ребёр по одному концу (любому из двух) и удалим из графа, то тем самым с помощью $\le \lambda$ удалённых вершин (поскольку одна и та же вершина могла встретиться дважды) мы сделаем граф несвязным. Таким образом, $\kappa \le \lambda$.

Докажем второе неравенство: $\lambda \le \delta$. Рассмотрим вершину минимальной степени, тогда мы можем удалить все $\delta$ смежных с ней рёбер и тем самым отделить эту вершину от всего остального графа. Следовательно, $\lambda \le \delta$.

Интересно, что неравенство Уитни \bf{нельзя улучшить}: т.е. для любых троек чисел, удовлетворяющих этому неравенству, существует хотя бы один соответствующий граф. См. задачу \algohref=connectivity_back_problem{"Построение графа с указанными величинами вершинной и рёберной связностей и наименьшей из степеней вершин"}.


\h2{ Нахождение вершинной связности }

Переберём пару вершин $s$ и $t$, и найдём минимальное количество вершин, которое надо удалить, чтобы разделить $s$ и $t$.

Для этого \bf{раздвоим} каждую вершину: т.е. у каждой вершины $i$ создадим по две копии --- одна $i_1$ для входящих рёбер, другая $i_2$ --- для выходящих, и эти две копии связаны друг с другом ребром $(i_1, i_2)$.

Каждое ребро $(u,v)$ исходного графа в этой модифицированной сети превратится в два ребра: $(u_2, v_1)$ и $(v_2, u_1)$.

Всем рёбрам проставим пропускную способность, равную единице. Найдём теперь максимальный поток в этом графе между истоком $s$ и стоком $t$. По построению графа, он и будет являться минимальным количеством вершин, необходимых для разделения $s$ и $t$.

Таким образом, если для поиска максимального потока мы выберем алгоритм \algohref=edmonds_karp{Эдмондса-Карпа}, работающий за время $O (n m^2)$, то общая асимптотика алгоритма составит $O (n^3 m^2)$. Впрочем, константа, скрытая в асимптотике, весьма мала: поскольку сделать граф, на котором алгоритмы бы работали долго при любой паре исток-сток, практически невозможно.

