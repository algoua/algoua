\h1{Центры тяжести многоугольников и многогранников}

\bf{Центром тяжести} (или \bf{центром масс}) некоторого тела называется точка, обладающая тем свойством, что если подвесить тело за эту точку, то оно будет сохранять свое положение.

Ниже рассмотрены двумерные и трёхмерные задачи, связанные с поиском различных центров масс --- в основном с точки зрения вычислительной геометрии.

В рассмотренных ниже решениях можно выделить два основных \bf{факта}. Первый --- что центр масс системы материальных точек равен среднему их координат, взятых с коэффициентами, пропорциональными их массам. Второй факт --- что если мы знаем центры масс двух непересекающихся фигур, то центр масс их объединения будет лежать на отрезке, соединяющем эти два центра, причём он будет делить его в то же отношении, как масса второй фигуры относится к массе первой.


\h2{Двумерный случай: многоугольники}

На самом деле, говоря о центре масс двумерной фигуры, можно иметь в виду одну из трёх следующих \bf{задач}:

\ul{
\li Центр масс системы точек --- т.е. вся масса сосредоточена только в вершинах многоугольника.
\li Центр масс каркаса --- т.е. масса многоугольника сосредоточена на его периметре.
\li Центр масс сплошной фигуры --- т.е. масса многоугольника распределена по всей его площади.
}

Каждая из этих задач имеет самостоятельное решение, и будет рассмотрена ниже отдельно.


\h3{Центр масс системы точек}

Это самая простая из трёх задач, и её решение --- известная физическая формула центра масс системы материальных точек:

$$ \vec{r_c} = \frac{ \sum\limits_i \vec{r_i} ~ m_i }{ \sum\limits_i m_i }, $$

где $m_i$ --- массы точек, $\vec{r_i}$ --- их радиус-векторы (задающие их положение относительно начала координат), и $\vec{r_c}$ --- искомый радиус-вектор центра масс.

В частности, если все точки имеют одинаковую массу, то координаты центра масс есть \bf{среднее арифметическое} координат точек. Для \bf{треугольника} эта точка называется \bf{центроидом} и совпадает с точкой пересечения медиан:

$$ \vec{r_c} = \frac{ \vec{r_1} + \vec{r_2} + \vec{r_3} }{ 3 }. $$

Для \bf{доказательства} этих формул достаточно вспомнить, что равновесие достигается в такой точке $r_c$, в которой сумма моментов всех сил равна нулю. В данном случае это превращается в условие того, чтобы сумма радиус-векторов всех точек относительно точки $r_c$, домноженных на массы соответствующих точек, равнялась нулю:

$$ \sum\limits_i \left( \vec{r_i} - \vec{r_c} \right) m_i = \vec{0}, $$

и, выражая отсюда $\vec{r_c}$, мы и получаем требуемую формулу.


\h3{Центр масс каркаса}

Будем считать для простоты, что каркас однороден, т.е. его плотность везде одна и та же.

Но тогда каждую сторону многоугольника можно заменить одной точкой --- серединой этого отрезка (т.к. центр масс однородного отрезка есть середина этого отрезка), с массой, равной длине этого отрезка.

Теперь мы получили задачу о системе материальных точек, и применяя к ней решение из предыдущего пункта, мы находим:

$$ \vec{r_c} = \frac{ \sum\limits_i \vec{r_i^\prime} ~ l_i }{ P }, $$

где $\vec{r_i^\prime}$ --- точка-середина $i$-ой стороны многоугольника, $l_i$ --- длина $i$-ой стороны, $P$ --- периметр, т.е. сумма длин сторон.

Для \bf{треугольника} можно показать следующее утверждение: эта точка является \bf{точкой пересечения биссектрис} треугольника, образованного серединами сторон исходного треугольника. (чтобы показать это, надо воспользоваться приведённой выше формулой, и затем заметить, что биссектрисы делят стороны получившегося треугольника в тех же соотношениях, что и центры масс этих сторон).


\h3{Центр масс сплошной фигуры}

Мы считаем, что масса распределена по фигуре однородно, т.е. плотность в каждой точке фигуры равна одному и тому же числу.

\h4{Случай треугольника}

Утверждается, что для треугольника ответом будет всё тот же \bf{центроид}, т.е. точка, образованная средним арифметическим координат вершин:

$$ \vec{r_c} = \frac{ \vec{r_1} + \vec{r_2} + \vec{r_3} }{ 3 }. $$

\h4{Случай треугольника: доказательство}

Приведём здесь элементарное доказательство, не использующее теорию интегралов. 

Первым подобное, чисто геометрическое, доказательство привёл Архимед, но оно было весьма сложным, с большим числом геометрических построений. Приведённое здесь доказательство взято из статьи Apostol, Mnatsakanian "Finding Centroids the Easy Way".

Доказательство сводится к тому, чтобы показать, что центр масс треугольника лежит на одной из медиан; повторяя этот процесс ещё дважды, мы тем самым покажем, что центр масс лежит в точке пересечения медиан, которая и есть центроид.

Разобьём данный треугольник $T$ на четыре, соединив середины сторон, как показано на рисунке:

\img{centroids_1.jpg}

Четыре получившихся треугольника подобны треугольнику $T$ с коэффициентом $1/2$.

Треугольники №1 и №2 вместе образуют параллелограмм, центр масс которого $c_{12}$ лежит в точке пересечения его диагоналей (поскольку это фигура, симметричная относительно обеих диагоналей, а, значит, её центр масс обязан лежать на каждой из двух диагоналей). Точка $c_{12}$ находится посередине общей стороны треугольников №1 и №2, а также лежит на медиане треугольника $T$:

\img{centroids_2.jpg}

Пусть теперь вектор $\vec{r}$ --- вектор, проведённый из вершины $A$ к центру масс $c_1$ треугольника №1, и пусть вектор $\vec{m}$ --- вектор, проведённый из $A$ к точке $c_{12}$ (которая, напомним, является серединой стороны, на которой она лежит):

\img{centroids_3.jpg}

Наша цель --- показать, что вектора $\vec{r}$ и $\vec{m}$ коллинеарны.

Обозначим через $c_3$ и $c_4$ точки, являющиеся центрами масс треугольников №3 и №4. Тогда, очевидно, центром масс совокупности этих двух треугольников будет точка $c_{34}$, являющаяся серединой отрезка $c_3 c_4$. Более того, вектор от точки $c_{12}$ к точке $c_{34}$ совпадает с вектором $\vec{r}$.

Искомый центр масс $c$ треугольника $T$ лежит посередине отрезка, соединяющего точки $c_{12}$ и $c_{34}$ (поскольку мы разбили треугольник $T$ на две части равных площадей: №1-№2 и №3-№4):

\img{centroids_4.jpg}

Таким образом, вектор от вершины $A$ к центроиду $c$ равен $\vec{m} + \vec{r}/2$. С другой стороны, т.к. треугольник №1 подобен треугольнику $T$ с коэффициентом $1/2$, то этот же вектор равен $2 \vec{r}$. Отсюда получаем уравнение:

$$ \vec{m} + \vec{r}/2 = 2 \vec{r}, $$

откуда находим:

$$ \vec{r} = \frac{2}{3} \vec{m}. $$

Таким образом, мы доказали, что вектора $\vec{r}$ и $\vec{m}$ коллинеарны, что и означает, что искомый центроид $c$ лежит на медиане, исходящей из вершины $A$.

Более того, попутно мы доказали, что центроид делит каждую медиану в отношении $2:1$, считая от вершины.



\h4{Случай многоугольника}

Перейдём теперь к общему случаю --- т.е. к случаю \bf{мноугоугольника}. Для него такие рассуждения уже неприменимы, поэтому сведём задачу к треугольной: а именно, разобьём многоугольник на треугольники (т.е. триангулируем его), найдём центр масс каждого треугольника, а затем найдём центр масс получившихся центров масс треугольников.

Окончательная формула получается следующей:

$$ \vec{r_c} = \frac{ \sum\limits_i \vec{r_i^\circ} ~ S_i }{ S }, $$

где $\vec{r_i^\circ}$ --- центроид $i$-го треугольника в триангуляции заданного многоугольника, $S_i$ --- площадь $i$-го треугольника триангуляции, $S$ --- площадь всего многоугольника.

Триангуляция выпуклого многоугольника --- тривиальная задача: для этого, например, можно взять треугольники $(r_1,r_{i-1},r_i)$, где $i = 3 \ldots n$.

\h4{Случай многоугольника: альтернативный способ}

С другой стороны, применение приведённой формулы не очень удобно для \bf{невыпуклых многоугольников}, поскольку произвести их триангуляцию --- сама по себе непростая задача. Но для таких многоугольников можно придумать более простой подход. А именно, проведём аналогию с тем, как можно искать площадь произвольного многоугольника: выбирается произвольная точка $z$, а затем суммируются знаковые площади треугольников, образованных этой точкой и точками многоугольника: $S = |\sum_{i=1}^n S_{z,p_i,p_{i+1}}|$. Аналогичный приём можно применить и для поиска центра масс: только теперь мы будем суммировать центры масс треугольников $(z,p_i,p_{i+1})$, взятых с коэффициентами, пропорциональными их площадям, т.е. итоговая формула для центра масс такова:

$$ \vec{r_c} = \frac{ \sum\limits_i {\vec r}_{z,p_i,p_{i+1}}^\circ ~ S_{z,p_i,p_{i+1}} }{ S }, $$

где $z$ --- произвольная точка, $p_i$ --- точки многоугольника, ${\vec r}_{z,p_i,p_{i+1}}^\circ$ --- центроид треугольника $(z,p_i,p_{i+1})$, $S_{z,p_i,p_{i+1}}$ --- знаковая площадь этого треугольника, $S$ --- знаковая площадь всего многоугольника (т.е. $S = \sum_{i=1}^{n} S_{z,p_i,p_{i+1}}$).


\h2{Трёхмерный случай: многогранники}

Аналогично двумерному случаю, в 3D можно говорить сразу о четырёх возможных постановках задачи:

\ul{
\li Центр масс системы точек --- вершин многогранника.
\li Центр масс каркаса --- рёбер многогранника.
\li Центр масс поверхности --- т.е. масса распределена по площади поверхности многогранника.
\li Центр масс сплошного многогранника --- т.е. масса распределена по всему многограннику.
}


\h3{Центр масс системы точек}

Как и в двумерном случае, мы можем применить физическую формулу и получить тот же самый результат:

$$ \vec{r_c} = \frac{ \sum\limits_i \vec{r_i} ~ m_i }{ \sum\limits_i m_i }, $$

который в случае равных масс превращается в среднее арифметическое координат всех точек.


\h3{Центр масс каркаса многогранника}

Аналогично двумерному случаю, мы просто заменяем каждое ребро многогранника материальной точкой, расположенной посередине этого ребра, и с массой, равной длине этого ребра. Получив задачу о материальных точках, мы легко находим её решение как взвешенную сумму координат этих точек.


\h3{Центр масс поверхности многогранника}

Каждая грань поверхности многогранника --- двухмерная фигура, центр масс которой мы умеем искать. Найдя эти центры масс и заменив каждую грань её центром масс, мы получим задачу с материальными точками, которую уже легко решить.


\h3{Центр масс сплошного многогранника}

\h4{Случай тетраэдра}

Как и в двумерном случае, решим сначала простейшую задачу --- задачу для тетраэдра.

Утверждается, что центр масс тетраэдра совпадает с точкой пересечения его медиан (медианой тетраэдра называется отрезок, проведённый из его вершины в центр масс противоположной грани; таким образом, медиана тетраэдра проходит через вершину и через точку пересечения медиан треугольной грани).

Почему это так? Здесь верны рассуждения, аналогичные двумерному случаю: если мы рассечём тетраэдр на два тетраэдра с помощью плоскости, проходящей через вершину тетраэдра и какую-нибудь медиану противоположной грани, то оба получившихся тетраэдра будут иметь одинаковый объём (т.к. треугольная грань разобьётся медианой на два треугольника равной площади, а высота двух тетраэдров не изменится). Повторяя эти рассуждения несколько раз, получаем, что центр масс лежит на точке пересечения медиан тетраэдра.

Эта точка --- точка пересечения медиан тетраэдра --- называется его \bf{центроидом}. Можно показать, что она на самом деле имеет координаты, равные среднему арифметическому координат вершин тетраэдра:

$$ \vec{r_c} = \frac{ \vec{r_1} + \vec{r_2} + \vec{r_3} + \vec{r_4} }{ 4 }. $$

(это можно вывести из того факта, что центроид делит медианы в отношении $1:3$)

Таким образом, между случаями тетраэдра и треугольника принципиальной разницы нет: точка, равная среднему арифметическому вершин, является центром масс сразу в двух постановках задачи: и когда массы находится только в вершинах, и когда массы распределены по всей площади/объёму. На самом деле, этот результат обобщается на произвольную размерность: центр масс произвольного \bf{симплекса} (simplex) есть среднее арифметическое координат его вершин.

\h4{Случай произвольного многогранника}

Перейдём теперь к общему случаю --- случаю произвольного многогранника.

Снова, как и в двумерном случае, мы производим сведение этой задачи к уже решённой: разбиваем многогранник на тетраэдры (т.е. производим его тетраэдризацию), находим центр масс каждого из них, и получаем окончательный ответ на задачу в виде взвешенной суммы найденных центров масс.

