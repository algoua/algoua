\h1{ Heavy-light декомпозиция }

\bf{Heavy-light декомпозиция} --- это достаточно общий приём, который позволяет эффективно решать многие задачи, сводящиеся к \bf{запросам на дереве}.

Простейший \bf{пример} задач такого вида --- это следующая задача. Дано дерево, каждой вершине которого приписано какое-то число. Поступают запросы вида $(a,b)$, где $a$ и $b$ --- номера вершин дерева, и требуется узнать максимальное число на пути между вершинами $a$ и $b$.


\h2{ Описание алгоритма }

Итак, пусть дано дерево $G$ с $n$ вершинами, подвешенное за некоторый корень.

Суть этой декомпозиции в том, чтобы \bf{разбить дерево на несколько путей} таким образом, чтобы для любой вершины $v$ получалось, что если мы будем подниматься от $v$ к корню, то по пути сменим не более $\log n$ путей. Кроме того, все пути должны не пересекаться друг с другом по рёбрам.

Понятно, что если мы научимся искать такую декомпозицию для любого дерева, это позволит свести любой запрос вида "узнать что-то на пути из $a$ в $b$" к нескольким запросам вида "узнать что-то на отрезке $[l;r]$ $k$-го пути".


\h3{ Построение heavy-light декомпозиции }

Посчитаем для каждой вершины $v$ размер её поддерева $s(v)$ (т.е. это количество вершин в поддереве вершины $v$, включая саму вершину).

Далее, рассмотрим все рёбра, ведущие к сыновьям какой-либо вершины $v$. Назовём ребро $(v,c)$ \bf{тяжёлым}, если оно ведёт в вершину $c$ такую, что:

$$ s(c) \ge \frac{ s(v) }{ 2 } ~~~~ \Leftrightarrow ~~~~ {\rm edge} ~ (v,c) ~ {\rm is ~ heavy}. $$

Все остальные рёбра назовём \bf{лёгкими}. Очевидно, что из одной вершины $v$ вниз может исходить максимум одно тяжёлое ребро (т.к. в противном случае у вершины $v$ было бы два сына размера $s(v)/2$, что с учётом самой вершины $v$ даёт размер $2 \cdot s(v) / 2 + 1 > s(v)$, т.е. пришли к противоречию).

Теперь построим саму \bf{декомпозицию} дерева на непересекающиеся пути. Рассмотрим все вершины, из которых не выходит вниз ни одного тяжёлого ребра, и будем идти от каждой из них вверх, пока не дойдём до корня дерева или не пройдём лёгкое ребро. В результате мы получим несколько путей --- покажем, что это и есть искомые пути heavy-light декомпозиции.


\h3{ Доказательство корректности алгоритма }

Во-первых, заметим, что полученные алгоритмом пути будут \bf{непересекающимися}. В самом деле, если бы два каких-то пути имели бы общее ребро, это бы означало, что из какой-то вершины исходит вниз два тяжёлых ребра, чего быть не может.

Во-вторых, покажем, что спускаясь от корня дерева до произвольной вершины, мы \bf{сменим по пути не более $\log n$ путей}. В самом деле, проход вниз по лёгкому ребру уменьшает размер текущего поддерева более чем вдвое:

$$ s(c) < \frac{ s(v) }{ 2 } ~~~~ \Leftrightarrow ~~~~ {\rm edge} ~ (v,c) ~ {\rm is ~ light}. $$

Таким образом, мы не могли пройти более $\log n$ лёгких рёбер. Однако переходить с одного пути на другой мы можем только через лёгкое ребро (т.к. каждый путь, кроме заканчивающихся в корне, содержит лёгкое ребро в конце; а попасть сразу посередине пути мы не можем).

Следовательно, по пути от корня до любой вершины мы не можем сменить более $\log n$ путей, что и требовалось доказать.


\h2{ Применения при решении задач }

При решении задач иногда бывает удобнее рассматривать heavy-light как набор \bf{вершинно-непересекающихся} путей (а не рёберо-непересекающихся). Для этого достаточно из каждого пути исключить последнее ребро, если оно являются лёгким ребром --- тогда никакие свойства не нарушатся, но теперь каждая вершина будет принадлежать ровно одному пути.

Ниже мы рассмотрим несколько типичных задач, которые можно решать с помощью heavy-light декомпозиции.

Отдельно стоит обратить внимание на задачу \bf{сумма чисел на пути}, поскольку это пример задачи, которая может быть решена и более простыми техниками.


\h3{ Максимальное число на пути между двумя вершинами }

Дано дерево, каждой вершине которого приписано какое-то число. Поступают запросы вида $(a,b)$, где $a$ и $b$ --- номера вершин дерева, и требуется узнать максимальное число на пути между вершинами $a$ и $b$.

Построим заранее heavy-light декомпозицию. Над каждым получившимся путём построим \algohref=segment_tree{дерево отрезков для максимума}, что позволит искать вершину с максимальным приписанным числом в указанном сегменте указанного пути за $O (\log n)$. Хотя число путей в heavy-light декомпозиции может достигать $n-1$, суммарный размер всех путей есть величина $O(n)$, поэтому и суммарный размер деревьев отрезков также будет линейным.

Теперь, для того чтобы отвечать на поступивший запрос $(a,b)$ найдём наименьшего общего предка $l$ этих вершин (например, \algohref=lca_simpler{методом двоичного подъёма}). Теперь задача свелась к двум запросам: $(a,l)$ и $(b,l)$, на каждый из которых мы можем ответить таким образом: найдём, в каком пути лежит нижняя вершина, сделаем запрос к этому пути, перейдём в вершину-конец этого пути, снова определим, в каком мы пути оказались и сделаем запрос к нему, и так далее, пока не дойдём до пути, содержащего $l$.

Аккуратно следует быть со случаем, когда, например, $a$ и $l$ оказались в одном пути --- тогда запрос максимума к этому пути надо делать не на суффиксе, а на внутреннем подотрезке.

Таким образом, в процессе ответа на один подзапрос мы пройдём по $O (\log n)$ путям, в каждом из них сделав запрос максимума на суффиксе или на префиксе/подотрезке (запрос на префиксе/подотрезке мог быть только один раз).

Так мы получили решение за $O (\log^2 n)$ на один запрос.

Если ещё дополнительно предпосчитать на каждом пути максимумы на всех суффиксах, то получится решение за $O (n \log n)$ --- т.к. запрос максимума не на суффиксе случается только один раз, когда мы доходим до вершины $l$.


\h3{ Сумма чисел на пути между двумя вершинами }

Дано дерево, каждой вершине которого приписано какое-то число. Поступают запросы вида $(a,b)$, где $a$ и $b$ --- номера вершин дерева, и требуется узнать сумму чисел на пути между вершинами $a$ и $b$. Возможен вариант этой задачи, когда дополнительно бывают запросы изменения числа, приписанного той или иной вершине.

Хотя эту задачу можно решать с помощью heavy-light декомпозиции, построив над каждым путём дерево отрезков для суммы (или просто предпосчитав частичные суммы, если в задаче отсутствуют запросы изменения), эта задача может быть решена \bf{более простыми техниками}.

Если запросы модификации отсутствуют, то узнавать сумму на пути между двумя вершинами можно параллельно с поиском LCA двух вершин в \algohref=lca_simpler{алгоритме двоичного подъёма} --- для этого достаточно во время препроцессинга для LCA подсчитывать не только $2^k$-ых предков каждой вершины, но и сумму чисел на пути до этого предка.

Есть и принципиально другой подход к этой задаче --- рассмотреть эйлеров обход дерева, и построить дерево отрезков над ним. Этот алгоритм рассматривается в \algohref=tree_painting{статье с решением похожей задачи}. (А если запросы модификации отсутствуют --- то достаточно обойтись предпосчётом частичных сумм, без дерева отрезков.)

Оба этих способа дают относительно простые решения с асимптотикой $O (\log n)$ на один запрос.


\h3{ Перекраска рёбер пути между двумя вершинами }

Дано дерево, каждое ребро изначально покрашено в белый цвет. Поступают запросы вида $(a,b,c)$, где $a$ и $b$ --- номера вершин, $c$ --- цвет, что означает, что все рёбра на пути из $a$ в $b$ надо перекрасить в цвет $c$. Требуется после всех перекрашиваний сообщить, сколько в итоге получилось рёбер каждого цвета.

Решение --- просто сделать \algohref=segment_tree{дерево отрезков с покраской на отрезке} над набором путей heavy-light декомпозиции.

Каждый запрос перекраски на пути $(a,b)$ превратится в два подзапроса $(a,l)$ и $(b,l)$, где $l$ --- наименьший общий предок вершин $a$ и $b$ (найденный, например, \algohref=lca_simpler{алгоритмом двоичного подъёма}), а каждый из этих подзапросов --- в $O (\log n)$ запросов к деревьям отрезков над путями.

Итого получается решение с асимптотикой $O (\log^2 n)$ на один запрос.



\h2{ Задачи в online judges }

Список задач, которые можно решить, используя heavy-light декомпозицию:

\ul{

\li \href=http://acm.timus.ru/problem.aspx?space=1&num=1553{TIMUS #1553 \bf{"Caves and Tunnels"} ~~~~ [сложность: средняя]}

\li \href=http://ipsc.ksp.sk/contests/ipsc2009/real/problems/l.php{IPSC 2009 L \bf{"Let there be rainbows!"} ~~~~ [сложность: средняя]}

\li \href=http://www.spoj.pl/problems/QTREE3/{SPOJ #2798 \bf{"Query on a tree again!"} ~~~~ [сложность: средняя]}

\li \href=http://codeforces.ru/contest/117/problem/E{Codeforces Beta Round #88 E \bf{"Дерево или не дерево"} ~~~~ [сложность: высокая] }

}