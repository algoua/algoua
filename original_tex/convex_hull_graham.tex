<h1>Построение выпуклой оболочки обходом Грэхэма</h1>

<p>Даны N точек на плоскости. Построить их выпуклую оболочку, т.е. наименьший выпуклый многоугольник, содержащий все эти точки.</p>
<p>Мы рассмотрим метод <b>Грэхэма</b> (Graham) (предложен в 1972 г.) с улучшениями Эндрю (Andrew) (1979 г.). С его помощью можно построить выпуклую оболочку за время <b>O (N log N)</b> с использованием только операций сравнения, сложения и умножения. Алгоритм является асимптотически оптимальным (доказано, что не существует алгоритма с лучшей асимптотикой), хотя в некоторых задачах он неприемлем (в случае параллельной обработки или при online-обработке).</p>
<h2>Описание</h2>
<p>Алгоритм. Найдём самую левую и самую правую точки A и B (если таких точек несколько, то возьмём самую нижнюю среди левых, и самую верхнюю среди правых). Понятно, что и A, и B обязательно попадут в выпуклую оболочку. Далее, проведём через них прямую AB, разделив множество всех точек на верхнее и нижнее подмножества S1 и S2 (точки, лежащие на прямой, можно отнести к любому множеству - они всё равно не войдут в оболочку). Точки A и B отнесём к обоим множествам. Теперь построим для S1 верхнюю оболочку, а для S2 - нижнюю оболочку, и объединим их, получив ответ. Чтобы получить, скажем, верхнюю оболочку, нужно отсортировать все точки по абсциссе, затем пройтись по всем точкам, рассматривая на каждом шаге кроме самой точки две предыдущие точки, вошедшие в оболочку. Если текущая тройка точек образует не правый поворот (что легко проверить с помощью <algohref=oriented_area>Ориентированной площади</algohref>), то ближайшего соседа нужно удалить из оболочки. В конце концов, останутся только точки, входящие в выпуклую оболочку.</p>
<p>Итак, алгоритм заключается в сортировке всех точек по абсциссе и двух (в худшем случае) обходах всех точек, т.е. требуемая асимптотика O (N log N) достигнута.</p>
<h2>Реализация</h2>
<code>struct pt {
	double x, y;
};

bool cmp (pt a, pt b) {
	return a.x &lt; b.x || a.x == b.x && a.y &lt; b.y;
}

bool cw (pt a, pt b, pt c) {
	return a.x*(b.y-c.y)+b.x*(c.y-a.y)+c.x*(a.y-b.y) &lt; 0;
}

bool ccw (pt a, pt b, pt c) {
	return a.x*(b.y-c.y)+b.x*(c.y-a.y)+c.x*(a.y-b.y) > 0;
}

void convex_hull (vector&lt;pt> & a) {
	if (a.size() == 1)  return;
	sort (a.begin(), a.end(), &cmp);
	pt p1 = a[0],  p2 = a.back();
	vector&lt;pt> up, down;
	up.push_back (p1);
	down.push_back (p1);
	for (size_t i=1; i&lt;a.size(); ++i) {
		if (i==a.size()-1 || cw (p1, a[i], p2)) {
			while (up.size()>=2 && !cw (up[up.size()-2], up[up.size()-1], a[i]))
				up.pop_back();
			up.push_back (a[i]);
		}
		if (i==a.size()-1 || ccw (p1, a[i], p2)) {
			while (down.size()>=2 && !ccw (down[down.size()-2], down[down.size()-1], a[i]))
				down.pop_back();
			down.push_back (a[i]);
		}
	}
	a.clear();
	for (size_t i=0; i&lt;up.size(); ++i)
		a.push_back (up[i]);
	for (size_t i=down.size()-2; i>0; --i)
		a.push_back (down[i]);
}</code>