\h1{Поиск точек сочленения}

Пусть дан связный неориентированный граф. \bf{Точкой сочленения} (или точкой артикуляции, англ. "cut vertex" или "articulation point") называется такая вершина, удаление которой делает граф несвязным.

Опишем алгоритм, основанный на поиске в глубину, работающий за $O(n+m)$, где $n$ --- количество вершин, $m$ --- рёбер.


\h2{Алгоритм}

Запустим обход в глубину из произвольной вершины графа; обозначим её через $\rm root$. Заметим следующий \bf{факт} (который несложно доказать):

\ul{

\li Пусть мы находимся в обходе в глубину, просматривая сейчас все рёбра из вершины $v \ne {\rm root}$. Тогда, если текущее ребро $(v,to)$ таково, что из вершины $to$ и из любого её потомка в дереве обхода в глубину нет обратного ребра в какого-либо предка вершины $v$, то вершина $v$ является точкой сочленения. В противном случае, т.е. если обход в глубину просмотрел все рёбра из вершины $v$, и не нашёл удовлетворяющего вышеописанным условиям ребра, то вершина $v$ не является точкой сочленения. (В самом деле, мы этим условием проверяем, нет ли другого пути из $v$ в $to$)

\li Рассмотрим теперь оставшийся случай: $v = {\rm root}$. Тогда эта вершина является точкой сочленения тогда и только тогда, когда эта вершина имеет более одного сына в дереве обхода в глубину. (В самом деле, это означает, что, пройдя из $\rm root$ по произвольному ребру, мы не смогли обойти весь граф, откуда сразу следует, что $\rm root$ --- точка сочленения).

}

(Ср. формулировку этого критерия с формулировкой критерия для \algohref=bridge_searching{алгоритма поиска мостов}.)

Теперь осталось научиться проверять этот факт для каждой вершины эффективно. Для этого воспользуемся "временами входа в вершину", вычисляемыми \algohref=dfs{алгоритмом поиска в глубину}.

Итак, пусть $tin[v]$ --- это время захода поиска в глубину в вершину $v$. Теперь введём массив $fup[v]$, который и позволит нам отвечать на вышеописанные запросы. Время $fup[v]$ равно минимуму из времени захода в саму вершину $tin[v]$, времён захода в каждую вершину $p$, являющуюся концом некоторого обратного ребра $(v,p)$, а также из всех значений $fup[to]$ для каждой вершины $to$, являющейся непосредственным сыном $v$ в дереве поиска:

$$ fup[v] = \min \cases{
tin[v], & \cr
tin[p], & {\rm for all} (v,p){\rm\ --- back edge } \cr
fup[to], & {\rm for all} (v,to){\rm\ --- tree edge } \cr
} $$

(здесь "back edge" --- обратное ребро, "tree edge" --- ребро дерева)

Тогда, из вершины $v$ или её потомка есть обратное ребро в её предка тогда и только тогда, когда найдётся такой сын $to$, что $fup[to] < tin[v]$.

Таким образом, если для текущего ребра $(v,to)$ (принадлежащего дереву поиска) выполняется $fup[to] \ge tin[v]$, то вершина $v$ является точкой сочленения. Для начальной вершины $v = {\rm root}$ критерий другой: для этой вершины надо посчитать число непосредственных сыновей в дереве обхода в глубину.


\h2{Реализация}

Если говорить о самой реализации, то здесь нам нужно уметь различать три случая: когда мы идём по ребру дерева поиска в глубину, когда идём по обратному ребру, и когда пытаемся пойти по ребру дерева в обратную сторону. Это, соответственно, случаи $used[to]=false$, $used[to]=true ~ \&\& ~ to \ne parent$, и $to=parent$. Таким образом, нам надо передавать в функцию поиска в глубину вершину-предка текущей вершины.

\code
vector<int> g[MAXN];
bool used[MAXN];
int timer, tin[MAXN], fup[MAXN];

void dfs (int v, int p = -1) {
	used[v] = true;
	tin[v] = fup[v] = timer++;
	int children = 0;
	for (size_t i=0; i<g[v].size(); ++i) {
		int to = g[v][i];
		if (to == p)  continue;
		if (used[to])
			fup[v] = min (fup[v], tin[to]);
		else {
			dfs (to, v);
			fup[v] = min (fup[v], fup[to]);
			if (fup[to] >= tin[v] && p != -1)
				IS_CUTPOINT(v);
			++children;
		}
	}
	if (p == -1 && children > 1)
		IS_CUTPOINT(v);
}

int main() {
	int n;
	... чтение n и g ...
	
	timer = 0;
	for (int i=0; i<n; ++i)
		used[i] = false;
	dfs (0);
}
\endcode

Здесь константе $\rm MAXN$ должно быть задано значение, равное максимально возможному числу вершин во входном графе.

Функция ${\rm IS\_CUTPOINT}(v)$ в коде --- это некая функция, которая будет реагировать на то, что вершина $v$ является точкой сочленения, например, выводить эту вершины на экран (надо учитывать, что для одной и той же вершины эта функция может быть вызвана несколько раз).


\h2{Задачи в online judges}

Список задач, в которых требуется искать точки сочленения:

\ul{
\li \href=http://uva.onlinejudge.org/index.php?option=com_onlinejudge&Itemid=8&category=13&page=show_problem&problem=1140{UVA #10199 \bf{"Tourist Guide"} ~~~~ [сложность: низкая]}
\li \href=http://uva.onlinejudge.org/index.php?option=com_onlinejudge&Itemid=8&category=5&page=show_problem&problem=251{UVA #315 \bf{"Network"} ~~~~ [сложность: низкая]}
}


