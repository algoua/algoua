\h1{ Троичная сбалансированная система счисления }

Троичная сбалансированная система счисления --- это нестандартная позиционная система счисления. Основание системы равно $3$, однако она отличается от обычной троичной системы тем, что цифрами являются $-1, 0, 1$. Поскольку использовать $-1$ для одной цифры очень неудобно, то обычно принимают какое-то специальное обозначение. Условимся здесь обозначать минус единицу буквой $z$.

Например, число $5$ в троичной сбалансированной системе записывается как $1zz$, а число $-5$ --- как $z11$. Троичная сбалансированная система счисления позволяет записывать отрицательные числа без записи отдельного знака "минус". Троичная сбалансированная система позволяет дробные числа (например, $1/3$ записывается как $0.1$).

\h2{ Алгоритм перевода }

Научимся переводить числа в троичную сбалансированную систему.

Для этого надо сначала перевести число в троичную систему.

Ясно, что теперь нам надо избавиться от цифр $2$, для чего заметим, что $2 = 3 - 1$, т.е. мы можем заменить двойку в текущем разряде на $-1$, при этом увеличив следующий (т.е. слева от него в естественной записи) разряд на $1$. Если мы будем двигаться по записи справа налево и выполнять вышеописанную операцию (при этом в каких-то разрядах может происходить переполнение больше $3$, в таком случае, естественно, "сбрасываем" лишние тройки в старший разряд), то придём к троичной сбалансированной записи. Как нетрудно убедиться, то же самое правило верно и для дробных чисел.

Более изящно вышеописанную процедуру можно описать так. Мы берём число в троичной системе счисления, прибавляем к нему бесконечное число $\ldots 11111.11111 \ldots$, а затем от каждого разряда результата отнимаем единицу (уже безо всяких переносов).

Зная теперь алгоритм перевода из обычной троичной системы в сбалансированную, легко можно реализовать операции сложения, вычитания и деления --- просто сводя их к соответствующим операциям над троичными несбалансированными числами.