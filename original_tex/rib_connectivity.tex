\h1{ Рёберная связность. Свойства и нахождение }


\h2{ Определение }

Пусть дан неориентированный граф $G$ с $n$ вершинами и $m$ рёбрами.

\bf{Рёберной связностью} $\lambda$ графа $G$ называется наименьшее число рёбер, которое нужно удалить, чтобы граф перестал быть связным.

Например, для несвязного графа рёберная связность равна нулю. Для связного графа с единственным мостом рёберная связность равна единице.

Говорят, что множество $S$ рёбер \bf{разделяет} вершины $s$ и $t$, если при удалении этих рёбер из графа вершины $u$ и $v$ оказываются в разных компонентах связности.

Ясно, что рёберная связность графа равна минимуму от наименьшего числа рёбер, разделяющих две вершины $s$ и $t$, взятому среди всевозможных пар $(s,t)$.


\h2{ Свойства }


\h3{ Соотношение Уитни }

\bf{Соотношение Уитни (Whitney)} (1932 г.) между рёберной связностью $\lambda$, \algohref=vertex_connectivity{вершинной связностью} $\kappa$ и наименьшей из степеней вершин $\delta$:

$$ \kappa \le \lambda \le \delta. $$

\bf{Докажем} это утверждение.

Докажем сначала первое неравенство: $\kappa \le \lambda$. Рассмотрим этот набор из $\lambda$ рёбер, делающих граф несвязным. Если мы возьмём от каждого из этих ребёр по одному концу (любому из двух) и удалим из графа, то тем самым с помощью $\le \lambda$ удалённых вершин (поскольку одна и та же вершина могла встретиться дважды) мы сделаем граф несвязным. Таким образом, $\kappa \le \lambda$.

Докажем второе неравенство: $\lambda \le \delta$. Рассмотрим вершину минимальной степени, тогда мы можем удалить все $\delta$ смежных с ней рёбер и тем самым отделить эту вершину от всего остального графа. Следовательно, $\lambda \le \delta$.

Интересно, что неравенство Уитни \bf{нельзя улучшить}: т.е. для любых троек чисел, удовлетворяющих этому неравенству, существует хотя бы один соответствующий граф. См. задачу \algohref=connectivity_back_problem{"Построение графа с указанными величинами вершинной и рёберной связностей и наименьшей из степеней вершин"}.


\h3{ Теорема Форда-Фалкерсона }

\bf{Теорема Форда-Фалкерсона} (1956 г.):

Для любых двух вершин наибольшее число рёберно-непересекающихся цепей, соединяющих их, равно наименьшему числу рёбер, разделяющих эти вершины.


\h2{ Нахождение рёберной связности }


\h3{ Простой алгоритм на основе поиска максимального потока }

Этот способ основан на теореме Форда-Фалекрсона.

Мы должны перебрать все пары вершин $(s,t)$, и между каждой парой найти наибольшее число непересекающихся по рёбрам путей. Эту величину можно найти с помощью алгоритма максимального потока: мы делаем $s$ истоком, $t$ --- стоком, а пропускную способность каждого ребра кладём равной 1.

Таким образом, псевдокод алгоритма таков:

\code
int ans = INF;
for (int s=0; s<n; ++s)
	for (int t=s+1; t<n; ++t) {
		int flow = ... величина максимального потока из s в t ...
		ans = min (ans, flow);
	}
\endcode

Асимптотика алгоритма при использовании \edmonds_karp{алгоритма Эдмондса-Карпа нахождения максимального потока} получается $O (n^2 \cdot n m ^2) = O (n^3 m^2)$, однако следует заметить, что скрытая в асимптотике константа весьма мала, поскольку практически невозможно создать такой граф, чтобы алгоритм нахождения максимального потока работал медленно сразу при всех стоках и истоках.

Особенно быстро такой алгоритм будет работать на случайных графах.


\h3{ Специальный алгоритм }

Используя потоковую терминологию, данная задача --- это задача поиска \bf{глобального минимального разреза}.

Для её решения разработаны специальные алгоритмы. На данном сайте представлен один из которых --- \algohref=stoer_wagner_mincut{алгоритм Штор-Вагнера}, работающий за время $O (n^3)$ или $O (n m)$.



\h2{ Литература }

\ul{

\li Hassler Whitney. \bf{Congruent Graphs and the Connectivity of Graphs} [1932]

\li Фрэнк Харари. \bf{Теория графов} [2003]

}