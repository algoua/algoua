\h1{ Суффиксное дерево. Алгоритм Укконена }

Эта статья --- временная заглушка, и не содержит никаких описаний.

Описание алгоритма Укконена можно найти, например, в книге Гасфилда "Строки, деревья и последовательности в алгоритмах".


\h2{ Реализация алгоритма Укконена }

Этот алгоритм строит суффиксное дерево для данной строки длины $n$ за время $O (n \log k)$, где $k$ --- размер алфавита (если его считать константой, то асимптотика получается $O (n)$).

Входными данными для алгоритма являются строка $s$ и её длина $n$, которые передаются в виде глобальных переменных.

Основная функция --- $\rm build\_tree$, она строит суффиксное дерево. Дерево хранится в виде массива структур $\rm node$, где ${\rm node}[0]$ --- корень суффиксного дерева.

В целях простоты кода рёбра хранятся в тех же самых структурах: для каждой вершины в её структуре $\rm node$ записаны данные о ребре, входящем в неё из предка. Итого, в каждой $\rm node$ хранятся: $(l,r)$, определяющие метку $s[l..r-1]$ ребра в предка, $\rm par$ --- вершина-предок, $\rm link$ --- суффиксная ссылка, $\rm next$ --- список исходящих рёбер.

\code
string s;
int n;

struct node {
	int l, r, par, link;
	map<char,int> next;

	node (int l=0, int r=0, int par=-1)
		: l(l), r(r), par(par), link(-1) {}
	int len()  {  return r - l;  }
	int &get (char c) {
		if (!next.count(c))  next[c] = -1;
		return next[c];
	}
};
node t[MAXN];
int sz;

struct state {
	int v, pos;
	state (int v, int pos) : v(v), pos(pos)  {}
};
state ptr (0, 0);

state go (state st, int l, int r) {
	while (l < r)
		if (st.pos == t[st.v].len()) {
			st = state (t[st.v].get( s[l] ), 0);
			if (st.v == -1)  return st;
		}
		else {
			if (s[ t[st.v].l + st.pos ] != s[l])
				return state (-1, -1);
			if (r-l < t[st.v].len() - st.pos)
				return state (st.v, st.pos + r-l);
			l += t[st.v].len() - st.pos;
			st.pos = t[st.v].len();
		}
	return st;
}

int split (state st) {
	if (st.pos == t[st.v].len())
		return st.v;
	if (st.pos == 0)
		return t[st.v].par;
	node v = t[st.v];
	int id = sz++;
	t[id] = node (v.l, v.l+st.pos, v.par);
	t[v.par].get( s[v.l] ) = id;
	t[id].get( s[v.l+st.pos] ) = st.v;
	t[st.v].par = id;
	t[st.v].l += st.pos;
	return id;
}

int get_link (int v) {
	if (t[v].link != -1)  return t[v].link;
	if (t[v].par == -1)  return 0;
	int to = get_link (t[v].par);
	return t[v].link = split (go (state(to,t[to].len()), t[v].l + (t[v].par==0), t[v].r));
}

void tree_extend (int pos) {
	for(;;) {
		state nptr = go (ptr, pos, pos+1);
		if (nptr.v != -1) {
			ptr = nptr;
			return;
		}

		int mid = split (ptr);
		int leaf = sz++;
		t[leaf] = node (pos, n, mid);
		t[mid].get( s[pos] ) = leaf;

		ptr.v = get_link (mid);
		ptr.pos = t[ptr.v].len();
		if (!mid)  break;
	}
}

void build_tree() {
	sz = 1;
	for (int i=0; i<n; ++i)
		tree_extend (i);
}
\endcode


\h2{ Сжатая реализация }

Приведём также следующую компактную реализацию алгоритма Укконена, предложенную \href=http://codeforces.ru/profile/freopen{freopen}:

\code
const int N=1000000,INF=1000000000;
string a;
int t[N][26],l[N],r[N],p[N],s[N],tv,tp,ts,la;
 
void ukkadd (int c) {
	suff:;
	if (r[tv]<tp) {
		if (t[tv][c]==-1) { t[tv][c]=ts;  l[ts]=la;
			p[ts++]=tv;  tv=s[tv];  tp=r[tv]+1;  goto suff; }
		tv=t[tv][c]; tp=l[tv];
	}
	if (tp==-1 || c==a[tp]-'a') tp++; else {
		l[ts+1]=la;  p[ts+1]=ts;
		l[ts]=l[tv];  r[ts]=tp-1;  p[ts]=p[tv];  t[ts][c]=ts+1;  t[ts][a[tp]-'a']=tv;
		l[tv]=tp;  p[tv]=ts;  t[p[ts]][a[l[ts]]-'a']=ts;  ts+=2;
		tv=s[p[ts-2]];  tp=l[ts-2];
		while (tp<=r[ts-2]) {  tv=t[tv][a[tp]-'a'];  tp+=r[tv]-l[tv]+1;}
		if (tp==r[ts-2]+1)  s[ts-2]=tv;  else s[ts-2]=ts; 
		tp=r[tv]-(tp-r[ts-2])+2;  goto suff;
	}
}

void build() {
	ts=2;
	tv=0;
	tp=0;
	fill(r,r+N,(int)a.size()-1);
	s[0]=1;
	l[0]=-1;
	r[0]=-1;
	l[1]=-1;
	r[1]=-1;
	memset (t, -1, sizeof t);
	fill(t[1],t[1]+26,0);
	for (la=0; la<(int)a.size(); ++la)
		ukkadd (a[la]-'a');
}
\endcode

Тот же самый код, прокомментированный:

\code
const int N=1000000,    // максимальное число вершин в суффиксном дереве
	INF=1000000000; // константа "бесконечность"
string a;       // входная строка, для которой надо построить дерево
int t[N][26],   // массив переходов (состояние, буква)
	l[N],   // левая 
	r[N],   // и правая границы подстроки из a, отвечающие ребру, входящему в вершину
	p[N],   // предок вершины
	s[N],   // суффиксная ссылка
	tv,     // вершина текущего суффикса (если мы посередине ребра, то нижняя вершина ребра)
	tp,     // положение в строке соответствующее месту на ребре (от l[tv] до r[tv] включительно)
	ts,     // количество вершин
	la;     // текущий символ строки

void ukkadd(int c) { // дописать к дереву символ c
	suff:;      // будем приходить сюда после каждого перехода к суффиксу (и заново добавлять символ)
	if (r[tv]<tp) { // проверим, не вылезли ли мы за пределы текущего ребра
		// если вылезли, найдем следующее ребро. Если его нет - создадим лист и прицепим к дереву
		if (t[tv][c]==-1) {t[tv][c]=ts;l[ts]=la;p[ts++]=tv;tv=s[tv];tp=r[tv]+1;goto suff;}
		tv=t[tv][c];tp=l[tv]; // в противном случае просто перейдем к следующему ребру
	}
	if (tp==-1 || c==a[tp]-'a') tp++; else { // если буква на ребре совпадает с c то идем по ребру, а иначе
		// разделяем ребро на два. Посередине - вершина ts
		l[ts]=l[tv];r[ts]=tp-1;p[ts]=p[tv];t[ts][a[tp]-'a']=tv;
		// ставим лист ts+1. Он соответствует переходу по c.
		t[ts][c]=ts+1;l[ts+1]=la;p[ts+1]=ts;
		// обновляем параметры текущей вершины. Не забыть про переход от предка tv к ts.
		l[tv]=tp;p[tv]=ts;t[p[ts]][a[l[ts]]-'a']=ts;ts+=2;
		// готовимся к спуску: поднялись на ребро и перешли по суффиксной ссылке.
		// tp будет отмечать, где мы в текущем суффиксе.
		tv=s[p[ts-2]];tp=l[ts-2];
		// пока текущий суффикс не кончился, топаем вниз
		while (tp<=r[ts-2]) {tv=t[tv][a[tp]-'a'];tp+=r[tv]-l[tv]+1;}
		// если мы пришли в вершину, то поставим в нее суффиксную ссылку, иначе поставим в ts
		// (ведь на след. итерации мы создадим ts).
		if (tp==r[ts-2]+1) s[ts-2]=tv; else s[ts-2]=ts; 
		// устанавливаем tp на новое ребро и идем добавлять букву к суффиксу.
		tp=r[tv]-(tp-r[ts-2])+2;goto suff;
	}
}

void build() {
	ts=2;
	tv=0;
	tp=0;
	fill(r,r+N,(int)a.size()-1);
	// инициализируем данные для корня дерева
	s[0]=1;
	l[0]=-1;
	r[0]=-1;
	l[1]=-1;
	r[1]=-1;
	memset (t, -1, sizeof t);
	fill(t[1],t[1]+26,0);
	// добавляем текст в дерево по одной букве
	for (la=0; la<(int)a.size(); ++la)
		ukkadd (a[la]-'a');
}
\endcode



\h2{ Задачи в online judges }

Задачи, которые можно решить, используя суффиксное дерево:

\ul{

\li \href=http://uva.onlinejudge.org/index.php?option=onlinejudge&page=show_problem&problem=1620{UVA #10679 \bf{"I Love Strings!!!"} ~~~~ [сложность: средняя]}

}
