\h1{ Матрица Татта }

Матрица Татта --- это изящный подход к решению задачи о \bf{паросочетании} в произвольном (не обязательно двудольном) графе. Правда, в простом виде алгоритм не выдаёт сами рёбра, входящие в паросочетание, а только размер максимального паросочетания в графе.

Ниже мы сначала рассмотрим результат, полученный Таттом (Tutte) для проверки существования совершенного паросочетания (т.е. паросочетания, содержащего $n/2$ рёбер, и потому насыщающего все $n$ вершин). После этого мы рассмотрим результат, полученный позже Ловасом (Lovasz), который уже позволяет искать размер максимального паросочетания, а не только ограничивается случаем совершенного паросочетания. Затем приводится результат Рабина (Rabin) и Вазирани (Vazirani), которые указали алгоритм восстановления самого паросочетания (как набора входящих в него рёбер).


\h2{ Определение }

Пусть дан граф $G$ с $n$ вершинами ($n$ --- чётно).

Тогда \bf{матрицей Татта} (Tutte) называется следующая матрица $n \times n$:

$$ \pmatrix{
0 & x_{12} & x_{13} & \ldots & x_{1(n-1)} & x_{1n} \cr
-x_{12} & 0 & x_{23} & \ldots & x_{2(n-1)} & x_{2n} \cr
-x_{13} & -x_{23} & 0 & \ldots & x_{3(n-1)} & x_{3n} \cr
\vdots & \vdots & \vdots & \ddots & \vdots & \vdots \cr
-x_{1(n-1)} & -x_{2(n-1)} & -x_{3(n-1)} & \ldots & 0 & x_{(n-1)n} \cr
-x_{1n} & -x_{2n} & -x_{3n} & \ldots & -x_{(n-1)n} & 0 \cr
} $$

где $x_{ij}$ ($1 \le i < j \le n$) --- это либо независимая переменная, соответствующая ребру между вершинами $i$ и $j$, либо тождественный ноль, если ребра между этими вершинами нет.

Таким образом, в случае полного графа с $n$ вершинами матрица Татта содержит $n (n-1) / 2$ независимых переменных, если же в графе какие-то рёбра отсутствуют, то соответствующие элементы матрицы Татта превращаются в нули. Вообще, число переменных в матрице Татта совпадает с числом рёбер графа.

Матрица Татта антисимметрична (кососимметрична).


\h2{ Теорема Татта }

Рассмотрим определитель $\det(A)$ матрицы Татта. Это, вообще говоря, многочлен относительно переменных $x_{ij}$.

\bf{Теорема Татта} гласит: в графе $G$ существует совершенное паросочетание тогда и только тогда, когда многочлен $\det(A)$ не равен нулю тождественно (т.е. имеет хотя бы одно слагаемое с ненулевым коэффициентом). Напомним, что паросочетание называется совершенным, если оно насыщает все вершины, т.е. его мощность равна $n/2$.

Канадский математик Вильям Томас Татт (William Thomas Tutte) первым указал на тесную связь между паросочетаниями в графах и определителями матриц (1947 г.). Более простой вид этой связи позже обнаружил Эдмондс (Edmonds) в случае двудольных графов (1967 г.). Рандомизированные алгоритмы для нахождения величины максимального паросочетания и самих рёбер этого паросочетания были предложены позже, соответственно, Ловасом (Lovasz) (в 1979 г.), и Рабином (Rabin) и Вазирани (Vazirani) (в 1984 г.).


\h3{ Практическое применение: рандомизированный алгоритм }

Непосредственно применять теорему Татта даже в задаче проверки существования совершенного паросочетания нецелесообразно. Причиной этого является то, что при символьном вычислении определителя (т.е. в виде многочленов над переменными $x_{ij}$) промежуточные результаты являются многочленами, содержащими $O(n^2)$ переменных. Поэтому вычисление определителя матрицы Татта в символьном виде потребует неоправданно много времени.

Венгерский математик Ласло Ловас (Laszlo Lovasz) был первым, указавшим возможность применения здесь \bf{рандомизированного} алгоритма для упрощения вычислений.

Идея очень проста: заменим все переменные $x_{ij}$ случайными числами:

$$ x_{ij} = {\rm rand}(). $$

Тогда, если полином $\det(A)$ был тождественно нулевым, после такой замены он и будет оставаться нулевым; если же он был отличным от нуля, то при такой случайной числовой замене вероятность того, что он обратится в ноль, достаточно мала.

Понятно, что такой тест (подстановка случайных значений и вычисление определителя $\det(A)$) если и ошибается, то только в одну сторону: может сообщить об отсутствии совершенного паросочетания, когда на самом деле оно существует.

Мы можем повторить этот тест несколько раз, подставляя в качестве значений переменных новые случайные числа, и с каждым повторным запуском мы получаем всё большую уверенность в том, что тест выдал правильный ответ. На практике в большинстве случаев достаточно одного теста, чтобы определить, есть ли в графе совершенное паросочетание или нет; несколько таких тестов дают уже весьма высокую вероятность.

Для оценки \bf{вероятности ошибки} можно использовать лемму Шварца-Зиппеля (Schwartz–Zippel), которая гласит, что вероятность обращения в ноль ненулевого полинома $P$ $k$-ой степени при подстановке в качестве значений переменных случайных чисел, каждое из которых может принимать $s$ вариантов значения, --- эта вероятность удовлетворяет неравенству:

$$ {\rm Pr}[P(r_1,\ldots,r_k)=0] \le \frac{k}{s}. $$

Например, при использовании стандартной функции случайных чисел C++ $\rm rand()$ получаем, что эта вероятность при $n=300$ составляет около процента.

\bf{Асимптотика} решения получается равной $O(n^3)$ (с использованием, например, \algohref=determinant_gauss{алгоритма Гаусса}), умноженное на количество итераций теста. Стоит отметить, что по асимптотике такое решение значительно отстаёт от решения \algohref=matching_edmonds{алгоритмом Эдмондса сжатия цветков}, однако в некоторых случаях более предпочтительно из-за простоты реализации.

\bf{Восстановить} само совершенное паросочетание как набор рёбер является более сложной задачей. Самым простым, хотя и медленным, вариантом будет восстановление этого паросочетания по одному ребру: перебираем первое ребро ответа, выбираем его так, чтобы в оставшемся графе существовало совершенное паросочетание, и т.д.


\h3{ Доказательство теоремы Татта }

Чтобы хорошо понять доказательство этой теоремы, сначала рассмотрим более простой результат, --- полученный Эдмондсом для случая двудольных графов.


\h4{ Теорема Эдмондса }

Рассмотрим двудольный граф, в каждой доле которого по $n$ вершин. Составим матрицу $B$ $n \times n$, в которой, по аналогии с матрицей Татта, $b_{ij}$ является отдельной независимой переменной, если ребро $(i,j)$ присутствует в графе, и является тождественным нулём в противном случае.

Эта матрица похожа на матрицу Татта, однако матрица Эдмондса имеет вдвое меньшую разность, и каждому ребру здесь соответствует только одна ячейка матрицы.

Докажем следующую \bf{теорему}: определитель $\det(B)$ отличен от нуля тогда и только тогда, когда в двудольном графе существует совершенное паросочетание.

\bf{Доказательство}. Распишем определитель согласно его определению, как сумма по всем перестановкам:

$$ \det(B) = \sum_{\pi \in S_n} {\rm sgn}(\pi) \cdot b_{1,\pi_1} \cdot b_{2,\pi_2} \cdot \ldots \cdot b_{n,\pi_n}. $$

Заметим, что поскольку все ненулевые элементы матрицы $B$ --- различные независимые переменные, то в этой сумме все ненулевые слагаемые различны, а потому никаких сокращений в процессе суммирования не происходит. Осталось заметить, что любое ненулевое слагаемое в этой сумме означает непересекающийся по вершинам набор рёбер, т.е. некоторое совершенное паросочетание. И наоборот, любому совершенному паросочетанию соответствует ненулевое слагаемое в этой сумме. Вкупе с вышесказанным это доказывает теорему.


\h4{ Свойства антисимметричных матриц }

Для доказательства теоремы Татта необходимо воспользоваться несколькими известными фактами линейной алгебры о свойствах антисимметричных матриц.

\bf{Во-первых}, если антисимметричная матрица имеет нечётный размер, то её определитель всегда равен нулю (теорема Якоби (Jacobi)). Для этого достаточно заметить, что антисимметричная матрица удовлетворяет равенству $A^T = -A$, и теперь получаем цепочку равенств:

$$ \det(A) = \det(A^T) = \det(-A) = (-1)^n \det(A), $$

откуда и следует, что при нечётных $n$ определитель необходимо должен быть равен нулю.

\bf{Во-вторых}, оказывается, что в случае антисимметричных матриц чётного размера их определитель всегда можно записать как квадрат некоторого полинома относительно переменных-элементов этой матрицы (полином называется пфаффианом (pfaffian), а результат принадлежит Мьюру (Muir)):

$$ \det(A) = {\rm Pf}^2(A). $$

\bf{В-третьих}, этот пфаффиан представляет собой не произвольный многочлен, а сумму вида:

$$ {\rm Pf}(A) = \frac{ 1 }{ 2^n n! } \sum_{\pi \in S_n} {\rm sgn(\pi)} \cdot a_{\pi_1,\pi_2} \cdot a_{\pi_3,\pi_4} \cdot \ldots \cdot a_{\pi_{n-1},\pi_n}. $$

Таким образом, каждое слагаемое в пфаффиане --- это произведение таких $n/2$ элементов матрицы, что их индексы в совокупности представляют собой разбиение множества $n$ на $n/2$ пар. Перед каждым слагаемым имеется свой коэффициент, но его вид нас здесь не интересует.


\h4{ Доказательство теоремы Татта }

Воспользовавшись вторым и третьим свойством из предыдущего пункта, мы получаем, что определитель $\det(A)$ матрицы Татта представляет собой квадрат от суммы слагаемых такого вида, что каждое слагаемое --- произведение элементов матрицы, индексы которых не повторяются и покрывают все номера от $1$ до $n$. Таким образом, снова, как и в доказательстве теоремы Эдмондса, каждое ненулевое слагаемое этой суммы соответствует совершенному паросочетанию в графе, и наоборот.


\h2{ Теорема Ловаса: обобщение для поиска размера максимального паросочетания }


\h3{ Формулировка }

\bf{Ранг} матрицы Татта совпадает с удвоенной величиной \bf{максимального паросочетания} в данном графе.


\h3{ Применение }

Для применения этой теоремы на практике можно воспользоваться тем же самым приёмом рандомизации, что и в вышеописанном алгоритме для матрицы Татта, а именно: подставить вместо переменных случайные значения, и найти ранг полученной числовой матрицы. Ранг матрицы, опять же, ищется за $O (n^3)$ с помощью модифицированного алгоритма Гаусса, см. \algohref=matrix_rank{здесь}.

Впрочем, следует отметить, что приведённая выше лемма Шварца-Зиппеля неприменима в явном виде, и интуитивно кажется, что вероятность ошибки здесь становится выше. Однако утверждается (см. работы Ловаса (Lovasz)), что и здесь вероятность ошибки (т.е. того, что ранг полученной матрицы окажется меньше, чем удвоенный размер максимального паросочетания) не превосходит $\frac{n}{s}$ (где $s$, как и выше, обозначает размер множества, из которого выбираются случайные числа).


\h3{ Доказательство }

Доказательство будет вытекать из теоремы линейной алгебры, известной как \bf{теорема Фробениуса} (Frobenius). Пусть дана антисимметричная матрица $A$ размера $n \times n$, и пусть множества $S$ и $T$ --- любые два подмножества множества $\{ 1, \ldots, n \}$, причём размеры этих множеств совпадают и равны рангу матрицы $A$. Обозначим через $A_{XY}$ матрицу, полученную из матрицы $A$ только строками с номерами из множества $X$ и столбцами с номерами из множества $Y$ (где $X$ и $Y$ --- некоторые подмножества множества $\{ 1, \ldots, n \}$). Тогда выполняется:

$$ \det(A_{SS}) \cdot \det(A_{TT}) = \det(A_{ST}) \cdot \det(A_{TS}). $$

Покажем, как это свойство позволяет установить \bf{соответствие} между рангом матрицы $A$ Татта и величиной максимального паросочетания.

С одной стороны, рассмотрим в графе $G$ некоторое максимальное паросочетание, и обозначим множество насыщаемых им вершин через $U$. Тогда, согласно теореме Татта, определитель $\det(A_{UU})$ отличен от нуля. Следователь, ранг матрицы Татта --- как минимум $2|U|$, т.е. не меньше удвоенной величины максимального паросочетания.

Покажем теперь обратное неравенство. Обозначим ранг матрицы $A$ через $r$. Это означает, что нашлась такая подматрица $A_{ST}$, где $|S| = |T| = r$, определитель которой отличен от нуля. Легко заметить, что $A_{TS}$ также будет отлично от нуля. Но по приведённой выше теореме Фробениуса это означает, что обе матрицы $A_{SS}$ и $A_{TT}$ имеют ненулевой определитель. Отсюда следует, что $r$ чётно (потому что, как было отмечено выше, антисимметричная матрица нечётной размерности всегда имеет нулевой определитель). Таким образом, мы можем применить к подматрице $A_{SS}$ (или $A_{TT}$) теорему Татта. Следовательно, в подграфе, индуцированном множеством вершин $S$ (или множеством вершин $T$), имеется совершенное паросочетание (и величина его равна $r/2$). Таким образом, ранг матрицы Татта не может быть больше удвоенной величины максимального паросочетания.

Объединяя два доказанных неравенства, мы получаем утверждение теоремы: ранг матрицы Татта совпадает с удвоенной величиной максимального паросочетания.


\h2{ Алгоритм Рабина-Вазирани нахождения максимального паросочетания }

Этот алгоритм является дальнейшим обобщением двух предыдущих теорем, и позволяет, в отличие от них, выдавать не только величину максимального паросочетания, но и сами рёбра, входящие в него.


\h3{ Формулировка теоремы }

Пусть в графе существует совершенное паросочетание. Тогда его матрица Татта невырождена, т.е. $\det(A) \ne 0$. Сгенерируем по ней, как было описано выше, случайную числовую матрицу $B$. Тогда, с высокой вероятностью, $(B^{-1})_{ji} \ne 0$ тогда и только тогда, когда ребро $(i,j)$ входит в какое-либо совершенное паросочетание.

(Здесь через $B^{-1}$ обозначена матрица, обратная к $B$. Предполагается, что определитель матрицы $B$ отличен от нуля, поэтому обратная матрица существует.)


\h3{ Применение }

Эту теорему можно применять для восстановления самих рёбер максимального паросочетания. Сначала придётся выделить подграф, в котором содержится искомое максимальное паросочетание (это можно сделать параллельно с алгоритмом поиска ранга матрицы).

После этого задача сводится к поиску совершенного паросочетания по данной числовой матрице, полученной из матрицы Татта. Здесь мы уже применяем теорему Рабина-Вазирани, --- находим обратную матрицу (что можно сделать модифицированным алгоритмом Гаусса за $O (n^3)$), находим в ней любой ненулевой элемент, удаляем из графа, и повторяем процесс. Асимптотика такого решения будет не самой быстрой --- $O (n^4)$, зато взамен получаем простоту решения (по сравнению, например, с \algohref=matching_edmonds{алгоритмом Эдмондса сжатия цветков}).


\h3{ Доказательство теоремы }

Вспомним известную формулу для элементов обратной матрицы $B^{-1}$:

$$ (B^{-1})_{ji} = \frac{ {\rm adj}(B)_{i,j} }{ \det(B) }, $$

где через ${\rm adj}(B)_{i,j}$ обозначено алгебраическое дополнение, т.е. это число $(-1)^{i+j}$, умноженное на определитель матрицы, получаемой из $B$ удалением $i$-й строки и $j$-го столбца.

Отсюда сразу получаем, что элемент $(B^{-1})_{ji}$ отличен от нуля тогда и только тогда, когда матрица $B$ с вычеркнутыми $i$-ой строкой и $j$-ым столбцом имеет ненулевой определитель, что, применяя теорему Татта, означает с высокой вероятностью, что в графе без вершин $i$ и $j$ по-прежнему существует совершенное паросочетание.


\h2{ Литература }

\ul{
\li \book{William Thomas Tutte}{The Factorization of Linear Graphs}{1946}{tutte.pdf}
\li \book{Laszlo Lovasz}{On Determinants, Matchings and Random Algorithms}{1979}
\li \book{Laszlo Lovasz, M.D. Plummer}{Matching Theory}{1986}{lovasz_plummer.pdf}
\li \book{Michael Oser Rabin, Vijay V. Vazirani}{Maximum matchings in general graphs through randomization}{1989}
\li \book{Allen B. Tucker}{Computer Science Handbook}{2004}{tucker.pdf}
\li \book{Rajeev Motwani, Prabhakar Raghavan}{Randomized Algorithms}{1995}{motwani.djvu}
\li \book{A.C. Aitken}{Determinants and matrices}{1944}{aitken.pdf}
}