\h1{ Код Грея }


\h2{ Определение }

Кодом Грея называется такая система нумерования неотрицательных чисел, когда коды двух соседних чисел отличаются ровно в одном бите.

Например, для чисел длины 3 бита имеем такую последовательность кодов Грея: $000$, $001$, $011$, $010$, $110$, $111$, $101$, $100$. Например, $G(4)=6$.

Этот код был изобретен Фрэнком Грэем (Frank Gray) в 1953 году.


\h2{ Нахождение кода Грея }

Рассмотрим биты числа $n$ и биты числа $G(n)$. Заметим, что $i$-ый бит $G(n)$ равен единице только в том случае, когда $i$-ый бит $n$ равен единице, а $i+1$-ый бит равен нулю, или наоборот ($i$-ый бит равен нулю, а $i+1$-ый равен единице). Таким образом, имеем: $G(n) = n \oplus (n>>1)$:

\code
int g (int n) {
	return n ^ (n >> 1);
}
\endcode


\h2{ Нахождение обратного кода Грея }

Требуется по коду Грея $g$ восстановить исходное число $n$.

Будем идти от старших битов к младшим (пусть самый младший бит имеет номер 1, а самый старший --- $k$). Получаем такие соотношения между битами $n_i$ числа $n$ и битами $g_i$ числа $g$:

$$ n_k = g_k, $$
$$ n_{k-1} = g_{k-1} \oplus n_k = g_k \oplus g_{k-1}, $$
$$ n_{k-2} = g_{k-2} \oplus n_{k-1} = g_k \oplus g_{k-1} \oplus g_{k-2}, $$
$$ n_{k-3} = g_{k-3} \oplus n_{k-2} = g_k \oplus g_{k-1} \oplus g_{k-2} \oplus g_{k-3}, $$
$$ \ldots $$

В виде программного кода это проще всего записать так:

\code
int rev_g (int g) {
	int n = 0;
	for (; g; g>>=1)
		n ^= g;
	return n;
}
\endcode


\h2{ Применения }

Коды Грея имеют несколько применений в различных областях, иногда достаточно неожиданных:

\ul{

\li $n$-битный код Грея соответствует гамильтонову циклу по $n$-мерному кубу.

\li В технике, коды Грея используются для \bf{минимизации ошибок} при преобразовании аналоговых сигналов в цифровые (например, в датчиках). В частности, коды Грея и были открыты в связи с этим применением.

\li Коды Грея применяются в решении задачи о \bf{Ханойских башнях}.

Пусть $n$ --- количество дисков. Начнём с кода Грея длины $n$, состоящего из одних нулей (т.е. $G(0)$), и будем двигаться по кодам Грея (от $G(i)$ переходить к $G(i+1)$). Поставим в соответствие каждому $i$-ому биту текущего кода Грея $i$-ый диск (причём самому младшему биту соответствует наименьший по размеру диск, а самому старшему биту --- наибольший). Поскольку на каждом шаге изменяется ровно один бит, то мы можем понимать изменение бита $i$ как перемещение $i$-го диска. Заметим, что для всех дисков, кроме наименьшего, на каждом шаге имеется ровно один вариант хода (за исключением стартовой и финальной позиций). Для наименьшего диска всегда имеется два варианта хода, однако имеется стратегия выбора хода, всегда приводящая к ответу: если $n$ нечётно, то последовательность перемещений наименьшего диска имеет вид $f \rightarrow t \rightarrow r \rightarrow f \rightarrow t \rightarrow r \rightarrow \ldots$ (где $f$ --- стартовый стержень, $t$ --- финальный стержень, $r$ --- оставшийся стержень), а если $n$ чётно, то $f \rightarrow r \rightarrow t \rightarrow f \rightarrow r \rightarrow t \rightarrow \ldots$.

\li Коды Грея также находят применение в теории \bf{генетических алгоритмов}.

}


\h2{ Задачи в online judges }

Список задач, которые можно сдать, используя коды Грея:

TODO