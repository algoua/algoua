\h1{ Кратчайшие пути фиксированной длины, количества путей фиксированной длины }

Ниже описываются решения этих двух задач, построенные на одной и той же идее: сведение задачи к возведению матрицы в степень (с обычной операцией умножения, и с модифицированной).


\h2{ Количество путей фиксированной длины }

Пусть задан ориентированный невзвешенный граф $G$ с $n$ вершинами, и задано целое число $k$. Требуется для каждой пары вершин $i$ и $j$ найти количество путей между этими вершинами, состоящих ровно из $k$ рёбер. Пути при этом рассматриваются произвольные, не обязательно простые (т.е. вершины могут повторяться сколько угодно раз).

Будем считать, что граф задан \bf{матрицей смежности}, т.е. матрицей $g[][]$ размера $n \times n$, где каждый элемент $g[i][j]$ равен единице, если между этими вершинами есть ребро, и нулю, если ребра нет. Описываемый ниже алгоритм работает и в случае наличия кратных рёбер: если между какими-то вершинами $i$ и $j$ есть сразу $m$ рёбер, то в матрицу смежности следует записать это число $m$. Также алгоритм корректно учитывает петли в графе, если таковые имеются.

Очевидно, что в таком виде \bf{матрица смежности} графа является \bf{ответом на задачу при $k=1$} --- она содержит количества путей длины $1$ между каждой парой вершин.

Решение будем строить \bf{итеративно}: пусть ответ для некоторого $k$ найден, покажем, как построить его для $k+1$. Обозначим через $d_k$ найденную матрицу ответов для $k$, а через $d_{k+1}$ --- матрицу ответов, которую необходимо построить. Тогда очевидна следующая формула:

$$ d_{k+1}[i][j] = \sum_{p = 1}^{n} d_k[i][p] \cdot g[p][j]. $$

Легко заметить, что записанная выше формула --- не что иное, как произведение двух матриц $d_k$ и $g$ в самом обычном смысле:

$$ d_{k+1} = d_k \cdot g. $$

Таким образом, \bf{решение} этой задачи можно представить следующим образом:

$$ d_k = \underbrace{ g \cdot \ldots \cdot g}_{k\ {\rm times}} = g^k. $$

Осталось заметить, что возведение матрицы в степень можно произвести эффективно с помощью алгоритма \algohref=binary_pow{\bf{Бинарного возведения в степень}}.

Итак, полученное решение имеет асимптотику $O (n^3 \log k)$ и заключается в бинарном возведении в $k$-ую степень матрицы смежности графа.


\h2{ Кратчайшие пути фиксированной длины }

Пусть задан ориентированный взвешенный граф $G$ с $n$ вершинами, и задано целое число $k$. Требуется для каждой пары вершин $i$ и $j$ найти длину кратчайшего пути между этими вершинами, состоящего ровно из $k$ рёбер.

Будем считать, что граф задан \bf{матрицей смежности}, т.е. матрицей $g[][]$ размера $n \times n$, где каждый элемент $g[i][j]$ содержит длину ребра из вершины $i$ в вершину $j$. Если между какими-то вершинами ребра нет, то соответствующий элемент матрицы считаем равным бесконечности $\infty$.

Очевидно, что в таком виде \bf{матрица смежности} графа является \bf{ответом на задачу при $k=1$} --- она содержит длины кратчайших путей между каждой парой вершин, или $\infty$, если пути длины $1$ не существует.

Решение будем строить \bf{итеративно}: пусть ответ для некоторого $k$ найден, покажем, как построить его для $k+1$. Обозначим через $d_k$ найденную матрицу ответов для $k$, а через $d_{k+1}$ --- матрицу ответов, которую необходимо построить. Тогда очевидна следующая формула:

$$ d_{k+1}[i][j] = \min_{p = 1 \ldots n} ( d_k[i][p] + g[p][j] ). $$

Внимательно посмотрев на эту формулу, легко провести аналогию с матричным умножением: фактически, матрица $d_k$ умножается на матрицу $g$, только в операции умножения вместо суммы по всем $p$ берётся минимум по всем $p$:

$$ d_{k+1} = d_k \odot g, $$

где операция $\odot$ умножения двух матриц определяется следующим образом:

$$ A \odot B = C \ \ \Longleftrightarrow\ \  C_{ij} = \min_{p=1 \ldots n} (A_{ip} + B_{pj}). $$

Таким образом, \bf{решение} этой задачи можно представить с помощью этой операции умножения следующим образом:

$$ d_k = \underbrace{ g \odot \ldots \odot g}_{k\ {\rm times}} = g^{\odot k}. $$

Осталось заметить, что возведение в степень с этой операцией умножения можно произвести эффективно с помощью алгоритма \algohref=binary_pow{\bf{Бинарного возведения в степень}}, поскольку единственное требуемое для него свойство --- ассоциативность операции умножения --- очевидно, имеется.

Итак, полученное решение имеет асимптотику $O (n^3 \log k)$ и заключается в бинарном возведении в $k$-ую степень матрицы смежности графа с изменённой операцией умножения матриц.


\h2{ Обобщение на случай, когда требуются пути длины, не более чем заданная длина }

Описанные выше решения решают задачи, когда требуется рассматривать пути определённой, фиксированной длины. Однако эти же решения можно приспособить и для решения задач, когда требуется рассматривать пути, содержащие \bf{не более} чем заданное число рёбер.

Сделать это можно, немного модифицировав входной граф. Например, если нас интересуют только пути, заканчивающиеся в определённой вершине $t$, то в граф можно \bf{добавить петлю} $(t,t)$ нулевого веса.

Если же нас по-прежнему интересуют ответы для всех пар вершин, то простое добавление петель ко всем вершинам испортит ответ. Вместо этого можно \bf{раздвоить} каждую вершину: для каждой вершины $v$ создать дополнительную вершину $v'$, провести ребро $(v,v')$ и добавить петлю $(v',v')$.

Решив на модифицированном графе задачу о поиске путей фиксированной длины, ответы на исходную задачу будут получаться как ответы между вершинами $i$ и $j'$ (т.е. дополнительные вершины --- это вершины-окончания, в которых мы можем "покрутиться" нужное число раз).


