\h1{Ожерелья}

Задача "ожерелья" --- это одна из классических комбинаторных задач. Требуется посчитать количество различных ожерелий из $n$ бусинок, каждая из которых может быть покрашена в один из $k$ цветов. При сравнении двух ожерелий их можно поворачивать, но не переворачивать (т.е. разрешается сделать циклический сдвиг).

\h2{Решение}

Решить эту задачу можно, используя \algohref=burnside_polya{лемму Бернсайда и теорему Пойа}. [ Ниже идёт копия текста из этой статьи ]

В этой задаче мы можем сразу найти группу инвариантных перестановок. Очевидно, она будет состоять из $n$ перестановок:

$$ \pi_0 = 1\ 2\ 3\ \ldots\ n $$
$$ \pi_1 = 2\ 3\ \ldots\ n\ 1 $$
$$ \pi_2 = 3\ \ldots\ n\ 1\ 2 $$
$$ \ldots $$
$$ \pi_{n-1} = n\ 1\ 2\ \ldots\ (n-1) $$

Найдём явную формулу для вычисления $C(\pi_i)$. Во-первых, заметим, что перестановки имеют такой вид, что в $i$-ой перестановке на $j$-ой позиции стоит $i+j$ (взятое по модулю $n$, если оно больше $n$). Если мы будем рассматривать циклическую структуру $i$-ой перестановки, то увидим, что единица переходит в $1+i$, $1+i$ переходит в $1+2i$, $1+2i$ --- в $1+3i$, и т.д., пока не придём в число $1 + kn$; для остальных элементов выполняются похожие утверждения. Отсюда можно понять, что все циклы имеют одинаковую длину, равную ${\rm lcm}(i,n) / i$, т.е. $n / {\rm gcd}(i,n)$ ("gcd" --- наибольший общий делитель, "lcm" --- наименьшее общее кратное). Тогда количество циклов в $i$-ой перестановке будет равно просто ${\rm gcd}(i,n)$.

Подставляя найденные значения в теорему Пойа, получаем \bf{решение}:

$$ {\rm Ans} = \frac{1}{n} \sum_{i=1}^{n} k ^ {{\rm gcd}(i,n)} $$

Можно оставить формулу в таком виде, а можно её свернуть ещё больше. Перейдём от суммы по всем $i$ к сумме только по делителям $n$. Действительно, в нашей сумме будет много одинаковых слагаемых: если $i$ не является делителем $n$, то таковой делитель найдётся после вычисления ${\rm gcd}(i,n)$. Следовательно, для каждого делителя $d|n$ его слагаемое $k^{{\rm gcd}(d,n)} = k^d$ учтётся несколько раз, т.е. сумму можно представить в таком виде:

$$ {\rm Ans} = \frac{1}{n} \sum_{d|n} C_d k^d $$

где $C_d$ --- это количество таких чисел $i$, что ${\rm gcd}(i,n) = d$. Найдём явное выражение для этого количества. Любое такое число $i$ имеет вид: $i=dj$, где ${\rm gcd}(j,n/d) = 1$ (иначе было бы ${\rm gcd}(i,n) > d$). Вспоминая \algohref=euler_function{функцию Эйлера}, мы находим, что количество таких $j$ --- это величина функции Эйлера $\phi(n/d)$. Таким образом, $C_d = \phi(n/d)$, и окончательно получаем \bf{формулу}:

$$ {\rm Ans} = \frac{1}{n} \sum_{d|n} \phi \left( \frac{n}{d} \right) k^d $$
