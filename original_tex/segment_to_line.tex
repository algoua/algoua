\h1{ Нахождение уравнения прямой для отрезка }

Задача --- по заданным координатам конца отрезка построить прямую, проходящую через него.

Мы считаем, что отрезок невырожден, т.е. имеет длину больше нуля (иначе, понятно, через него проходит бесконечно много различных прямых).


\h2{ Двумерный случай }

Пусть дан отрезок $PQ$, т.е. известны координаты его концов $P_x$, $P_y$, $Q_x$, $Q_y$.

Требуется построить \bf{уравнение прямой на плоскости}, проходящей через этот отрезок, т.е. найти коэффициенты $A$, $B$, $C$ в уравнении прямой:

$$ A x + B y + C = 0. $$

Заметим, что искомых троек $(A,B,C)$, проходящих через заданный отрезок, \bf{бесконечно много}: можно умножить все три коэффициента на произвольное ненулевое число и получить ту же самую прямую. Следовательно, наша задача --- найти одну из таких троек.

Нетрудно убедиться (подстановкой этих выражений и координат точек $P$ и $Q$ в уравнение прямой), что подходит следующий набор коэффициентов:

$$ A = P_y - Q_y, $$
$$ B = Q_x - P_x, $$
$$ C = - A P_x - B P_y. $$


\h3{ Целочисленный случай }

Важным преимуществом такого способа построения прямой является то, что если координаты концов были целочисленными, то и полученные коэффициенты также будут \bf{целочисленными}. В некоторых случаях это позволяет производить геометрические операции, вообще не прибегая к вещественным числам.

Однако есть и небольшой недостаток: для одной и той же прямой могут получаться разные тройки коэффициентов. Чтобы избежать этого, но не уходить от целочисленных коэффициентов, можно применить следующий приём, часто называемый \bf{нормированием}. Найдём \algohref=euclid_algorithm{наибольший общий делитель} чисел $|A|$, $|B|$, $|C|$, поделим на него все три коэффициента, а затем произведём нормировку знака: если $A<0$ или $A=0, B<0$, то умножим все три коэффициента на $-1$. В итоге мы придём к тому, что для одинаковых прямых будут получаться одинаковые тройки коэффициентов, что позволит легко проверять прямые на равенство.


\h3{ Вещественнозначный случай }

При работе с вещественными числами следует всегда помнить о погрешностях.

Коэффициенты $A$ и $B$ получаются у нас порядка исходных координат, коэффициент $C$ --- уже порядка квадрата от них. Это уже может быть достаточно большими числами, а, например, при \algohref=lines_intersection{пересечении прямых} они станут ещё больше, что может привести к большим ошибкам округления уже при исходных координатах порядка $10^3$.

Поэтому при работе с вещественными числами желательно производить так называемую \bf{нормировку} прямой: а именно, делать коэффициенты такими, чтобы $A^2 + B^2 = 1$. Для этого надо вычислить число $Z$:

$$ Z = \sqrt{ A^2 + B^2 }, $$

и разделить все три коэффициента $A$, $B$, $C$ на него.

Тем самым, порядок коэффициентов $A$ и $B$ уже не будет зависеть от порядка входных координат, а коэффициент $C$ будет того же порядка, что и входные координаты. На практике это приводит к значительному улучшению точности вычислений.

Наконец, упомянем о \bf{сравнении} прямых --- ведь после такой нормировки для одной и той же прямой могут получаться только две тройки коэффициентов: с точностью до умножения на $-1$. Соответственно, если мы произведём дополнительную нормировку с учётом знака (если $A<-\varepsilon$ или $|A|<\varepsilon, B<-\varepsilon$, то умножать на $-1$), то получающиеся коэффициенты будут уникальными.


\h2{ Трёхмерный и многомерный случай }

Уже в трёхмерном случае \bf{нет простого уравнения}, описывающего прямую (её можно задать как пересечение двух плоскостей, т.е. систему двух уравнений, но это неудобный способ).

Следовательно, в трёхмерном и многомерном случаях мы должны пользоваться \bf{параметрическим способом задания прямой}, т.е. в виде точки $p$ и вектора $v$:

$$ p + v t, ~~~ t \in \cal{R}. $$

Т.е. прямая --- это все точки, которые можно получить из точки $p$ прибавлением вектора $v$ с произвольным коэффициентом.

\bf{Построение} прямой в параметрическом виде по координатам концов отрезка --- тривиально, мы просто берём один конец отрезка за точку $p$, а вектор из первого до второго конца --- за вектор $v$.