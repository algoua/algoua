<h1>Минимальное остовное дерево. Алгоритм Крускала с системой непересекающихся множеств</h1>

<p>Постановку задачи и описание алгоритма Крускала см. <algohref=mst_kruskal>здесь</algohref>.</p>
<p>Здесь будет рассмотрена реализация с использованием структуры данных <algohref=dsu>"система непересекающихся множеств" (DSU)</algohref>, которая позволит достигнуть асимптотики <b>O (M log N)</b>.</p>
<h3>Описание</h3>
<p>Так же, как и в простой версии алгоритма Крускала, отсортируем все рёбра по неубыванию веса. Затем поместим каждую вершину в своё дерево (т.е. своё множество) с помощью вызова функции DSU MakeSet - на это уйдёт в сумме O (N). Перебираем все рёбра (в порядке сортировки) и для каждого ребра за O (1) определяем, принадлежат ли его концы разным деревьям (с помощью двух вызовов FindSet за O (1)). Наконец, объединение двух деревьев будет осуществляться вызовом Union - также за O (1). Итого мы получаем асимптотику O (M log N + N + M) = O (M log N).</p>
<h3>Реализация</h3>
<p>Для уменьшения объёма кода реализуем все операции не в виде отдельных функций, а прямо в коде алгоритма Крускала.</p>
<p>Здесь будет использоваться рандомизированная версия DSU.</p>
<code>vector&lt;int> p (n);

int dsu_get (int v) {
	return (v == p[v]) ? v : (p[v] = dsu_get (p[v]));
}

void dsu_unite (int a, int b) {
	a = dsu_get (a);
	b = dsu_get (b);
	if (rand() & 1)
		swap (a, b);
	if (a != b)
		p[a] = b;
}

... в функции main(): ...

int m;
vector &lt; pair &lt; int, pair&lt;int,int> > > g; // вес - вершина 1 - вершина 2
... чтение графа ...

int cost = 0;
vector &lt; pair&lt;int,int> > res;

sort (g.begin(), g.end());
p.resize (n);
for (int i=0; i&lt;n; ++i)
	p[i] = i;
for (int i=0; i&lt;m; ++i) {
	int a = g[i].second.first,  b = g[i].second.second,  l = g[i].first;
	if (dsu_get(a) != dsu_get(b)) {
		cost += l;
		res.push_back (g[i].second);
		dsu_unite (a, b);
	}
}</code>