\h1{ Нахождение наидлиннейшей возрастающей подпоследовательности }

\bf{Условие задачи} следующее. Дан массив из $n$ чисел: $a[0 \ldots n-1]$. Требуется найти в этой последовательности строго возрастающую подпоследовательность наибольшей длины.

\bf{Формально} это выглядит следующим образом: требуется найти такую последовательность индексов $i_1 \ldots i_k$, что:

$$ i_1 < i_2 < \ldots < i_k, $$
$$ a[i_1] < a[i_2] < \ldots < a[i_k]. $$

В данной статье рассматриваются различные алгоритмы решения данной задачи, а также некоторые задачи, которые можно свести к данной задаче.



\h2{ Решение за $O(n^2)$: метод динамического программирования }

Динамическое программирование --- это весьма общая методика, позволяющая решать огромный класс задач. Здесь мы рассмотрим эту методику применительно к нашей конкретной задаче.

Научимся сначала искать \bf{длину} наидлиннейшей возрастающей подпоследовательности, а восстановлением самой подпоследовательности займёмся чуть позже.


\h3{ Динамическое программирование для поиска длины ответа }

Для этого давайте научимся считать массив $d[0 \ldots n-1]$, где $d[i]$ --- это длина наидлиннейшей возрастающей подпоследовательности, оканчивающейся именно в элементе с индексом $i$. Массив этот (он и есть --- сама динамика) будем считать постепенно: сначала $d[0]$, затем $d[1]$ и т.д. В конце, когда этот массив будет подсчитан нами, ответ на задачу будет равен максимуму в массиве $d[]$.

Итак, пусть текущий индекс --- $i$, т.е. мы хотим посчитать значение $d[i]$, а все предыдущие значения $d[0] 
\ldots d[i-1]$ уже подсчитаны. Тогда заметим, что у нас есть два варианта:

\ul{

\li либо $d[i] = 1$, т.е. искомая подпоследовательность состоит только из числа $a[i]$.

\li либо $d[i] > 1$. Тогда перед числом $a[i]$ в искомой подпоследовательности стоит какое-то другое число. Давайте переберём это число: это может быть любой элемент $a[j]$ $(j = 0 \ldots i-1)$, но такой, что $a[j] < a[i]$. Пусть мы рассматриваем какой-то текущий индекс $j$. Поскольку динамика $d[j]$ для него уже подсчитана, получается, что это число $a[j]$ вместе с числом $a[i]$ даёт ответ $d[j] + 1$. Таким образом, $d[i]$ можно считать по такой формуле:

$$ d[i] = \max_{j=0 \ldots i-1, \atop a[j] < a[i]} ( d[j] + 1 ). $$

}

Объединяя эти два варианта в один, получаем окончательный алгоритм для вычисления $d[i]$:

$$ d[i] = \max \Big( 1, \max_{j=0 \ldots i-1, \atop a[j] < a[i]} ( d[j] + 1 ) \Big). $$

Этот алгоритм --- и есть сама динамика.


\h3{ Реализация }

Приведём реализацию описанного выше алгоритма, которая находит и выводит длину наидлиннейшей возрастающей подпоследовательности:

\code
int d[MAXN]; // константа MAXN равна наибольшему возможному значению n

for (int i=0; i<n; ++i) {
	d[i] = 1;
	for (int j=0; j<i; ++j)
		if (a[j] < a[i])
			d[i] = max (d[i], 1 + d[j]);
}

int ans = d[0];
for (int i=0; i<n; ++i)
	ans = max (ans, d[i]);
cout << ans << endl;
\endcode


\h3{ Восстановление ответа }

Пока мы лишь научились искать длину ответа, но саму наидлиннейшую подпоследовательность мы вывести не можем, т.к. не сохраняем никакой дополнительной информации о том, где достигаются максимумы.

Чтобы суметь восстановить ответ, помимо динамики $d[0 \ldots n-1]$ надо также хранить вспомогательный массив $p[0 \ldots n-1]$ --- то, в каком месте достигся максимум для каждого значения $d[i]$. Иными словами, индекс $p[i]$ будет обозначать тот самый индекс $j$, при котором получилось наибольшее значение $d[i]$. (Этот массив $p[]$ в динамическом программировании часто называют "массивом предков".)

Тогда, чтобы вывести ответ, надо просто идти от элемента с максимальным значением $d[i]$ по его предкам до тех пор, пока мы не выведем всю подпоследовательность, т.е. пока не дойдём до элемента со значением $d = 1$.


\h3{ Реализация восстановления ответа }

Итак, у нас изменится и код самой динамики, и добавится код, производящий вывод наидлиннейшей подпоследовательности (выводятся индексы элементов подпоследовательности, в 0-индексации).

Для удобства мы изначально положили индексы $p[i] = -1$: для элементов, у которых динамика получилась равной единице, это значение предка так и останется минус единицей, что чуть-чуть удобнее при восстановлении ответа.

\code
int d[MAXN], p[MAXN]; // константа MAXN равна наибольшему возможному значению n

for (int i=0; i<n; ++i) {
	d[i] = 1;
	p[i] = -1;
	for (int j=0; j<i; ++j)
		if (a[j] < a[i])
			if (1 + d[j] > d[i]) {
				d[i] = 1 + d[j];
				p[i] = j;
			}
}

int ans = d[0],  pos = 0;
for (int i=0; i<n; ++i)
	if (d[i] > ans) {
		ans = d[i];
		pos = i;
	}
cout << ans << endl;

vector<int> path;
while (pos != -1) {
	path.push_back (pos);
	pos = p[pos];
}
reverse (path.begin(), path.end());
for (int i=0; i<(int)path.size(); ++i)
	cout << path[i] << ' ';
\endcode


\h3{ Альтернативный способ восстановления ответа }

Впрочем, как почти всегда в случае динамического программирования, для восстановления ответа можно не хранить дополнительный массив предков $p[]$, а просто заново пересчитывая текущий элемент динамики и ища, на каком же индексе был достигнут максимум.

Этот способ при реализации приводит к чуть более длинному коду, однако взамен получаем экономию памяти и абсолютное совпадение логики программы в процессе подсчёта динамики и в процессе восстановления.



\h2{ Решение за $O (n \log n)$: динамическое программирование с двоичным поиском }

Чтобы получить более быстрое решение задачи, построим другой вариант динамического программирования за $O (n^2)$, а затем поймём, как можно этот вариант ускорить до $O (n \log n)$.

\bf{Динамика} теперь будет такой: пусть $d[i]$ $(i = 0 \ldots n)$ --- это число, на которое оканчивается возрастающая подпоследовательность длины $i$ (а если таких чисел несколько --- то наименьшее из них).

Изначально мы полагаем $d[0] = -\infty$, а все остальные элементы $d[i] = \infty$.

Считать эту динамику мы будем постепенно, обработав число $a[0]$, затем $a[1]$, и т.д.

Приведём реализацию этой динамики за $O (n^2)$:

\code
int d[MAXN];
d[0] = -INF;
for (int i=1; i<=n; ++i)
	d[i] = INF;

for (int i=0; i<n; i++)
	for (int j=1; j<=n; j++)
		if (d[j-1] < a[i] && a[i] < d[j])
			d[j] = a[i];
\endcode

Заметим теперь, что у этой динамики есть одно \bf{очень важное свойство}: $d[i-1] \le d[i]$ для всех $i = 1 \ldots n$. Другое свойство --- что каждый элемент $a[i]$ обновляет максимум одну ячейку $d[j]$.

Таким образом, это означает, что обрабатывать очередное $a[i]$ мы можем за $O (\log n)$, сделав двоичный поиск по массиву $d[]$. В самом деле, мы просто ищем в массиве $d[]$ первое число, которое строго больше $a[i]$, и пытаемся произвести обновление этого элемента аналогично приведённой выше реализации.


\h3{ Реализация за $O (n \log n)$ }

Воспользовавшись стандартным в языке C++ алгоритмом двоичного поиска $upper\_bound$ (который возвращает позицию первого элемента, строго большего данного), получаем такую простую реализацию:

\code
int d[MAXN];
d[0] = -INF;
for (int i=1; i<=n; ++i)
	d[i] = INF;

for (int i=0; i<n; i++) {
	int j = int (upper_bound (d.begin(), d.end(), a[i]) - d.begin());
	if (d[j-1] < a[i] && a[i] < d[j])
		d[j] = a[i];
}
\endcode


\h3{ Восстановление ответа }

По такой динамике тоже можно восстановить ответ, для чего опять же помимо динамики $d[i]$ также надо хранить массив "предков" $p[i]$ --- то, на элементе с каким индексом оканчивается оптимальная подпоследовательность длины $i$. Кроме того, для каждого элемента массива $a[i]$ надо будет хранить его "предка" --- т.е. индекс того элемента, который должен стоять перед $a[i]$ в оптимальной подпоследовательности.

Поддерживая эти два массива по ходу вычисления динамики, в конце будет нетрудно восстановить искомую подпоследовательность.

(Интересно отметить, что применительно к данной динамике ответ можно восстанавливать только так, через массивы предков --- а без них восстановить ответ после вычисления динамики будет невозможно. Это один из редких случаев, когда к динамике неприменим альтернативный способ восстановления --- без массивов предков).



\h2{ Решение за $O (n \log n)$: структуры данных }

Если приведённый выше способ за $O (n \log n)$ весьма красив, однако не совсем тривиален идейно, то есть и другой путь: воспользоваться одной из известных простых структур данных.

В самом деле, давайте вернёмся к самой первой динамике, где состоянием являлась просто текущая позиция. Текущее значение динамики $d[i]$ вычисляется как максимум значений $d[i]$ среди всех таких элементов $j$, что $a[j] < a[i]$.

Следовательно, если мы через $t[]$ обозначим такой \bf{массив}, в который будем записывать значения динамики от чисел:

$$ t[a[i]] = d[i], $$

то получается, что всё, что нам надо уметь --- это искать \bf{максимум на префиксе} массива $t$: $t[0 \ldots a[i]-1]$.

Задача поиска максимума на префиксах массива (с учётом того, что массив может меняться) решается многими стандартными структурами данных, например, \algohref=segment_tree{деревом отрезков} или \algohref=fenwick_tree{деревом Фенвика}.

Воспользовавшись любой такой структурой данных, мы получим решение за $O (n \log n)$.

У этого способа решения есть явные \bf{недостатки}: по длине и сложности реализации этот путь будет в любом случае хуже, чем описанная выше динамика за $O (n \log n)$. Кроме того, если входные числа $a[i]$ могут быть достаточно большими, то скорее всего их придётся сжимать (т.е. перенумеровывать от $0$ до $n-1$) --- без этого многие стандартные структуры данных работать не смогут из-за высокого потребления памяти.

С другой стороны, у данного пути есть и \bf{преимущества}. Во-первых, при таком способе решения не придётся задумываться о хитрой динамике. Во-вторых, этот способ позволяет решать некоторые обобщения нашей задачи (о них см. ниже).



\h2{ Смежные задачи }

Приведём здесь несколько задач, тесно связанных с задачей поиска наидлиннейшей возрастающей подпоследовательности.


\h3{ Наидлиннейшая неубывающая подпоследовательность }

Фактически, это та же самая задача, только теперь в искомой подпоследовательности допускаются одинаковые числа (т.е. мы должны найти нестрого возрастающую подпоследовательность).

Решение этой задачи по сути ничем не отличается от нашей исходной задачи, просто при сравнениях изменятся знаки неравенств, а также надо будет немного изменить двоичный поиск.


\h3{ Количество наидлиннейших возрастающих подпоследовательностей }

Для решения этой задачи можно использовать самую первую динамику за $O (n^2)$ либо подход с помощью структур данных для решения за $O (n \log n)$. И в том, и в том случае все изменения заключаются только в том, что помимо значения динамики $d[i]$ надо также хранить, сколькими способами это значение могло быть получено.

По всей видимости, способ решения через динамику за $O (n \log n)$ к данной задаче применить невозможно.


\h3{ Наименьшее число невозрастающих подпоследовательностей, покрывающих данную последовательность }

\bf{Условие} таково. Дан массив из $n$ чисел $a[0 \ldots n-1]$. Требуется раскрасить его числа в наименьшее число цветов так, чтобы по каждому цвету получалась бы невозрастающая подпоследовательность.

\bf{Решение}. Утверждается, что минимальное количество необходимых цветов равно длине наидлиннейшей возрастающей подпоследовательности.

\bf{Доказательство}. Фактически, нам надо доказать \bf{двойственность} этой задачи и задачи поиска наидлиннейшей возрастающей подпоследовательности.

Обозначим через $x$ длину наидлиннейшей возрастающей подпоследовательности, а через $y$ --- искомое наименьшее число невозрастающих подпоследовательностей. Нам надо доказать, что $x=y$.

С одной стороны, понятно, почему не может быть $y<x$: ведь если у нас есть $x$ строго возрастающих элементов, то никакие два из них не могли попасть в одну невозрастающую подпоследовательность, а, значит, $y \ge x$.

Покажем теперь, что, наоборот, $y$ не может быть $> x$. Докажем это от противного: предположим, что $y > x$. Тогда рассмотрим любой оптимальный набор из $y$ невозрастающих подпоследовательностей. Преобразуем этот набор таким образом: пока есть две таких подпоследовательности, что первая начинается раньше второй, но при этом первая начинается с числа, больше либо равного чем начало второй --- отцепим это стартовое число от первой подпоследовательности и прицепим в начало второй. Таким образом, через какое-то конечное число шагов у нас останется $y$ подпоследовательностей, причём их стартовые числа будут образовывать возрастающую подпоследовательность длины $y$. Но $y > x$, т.е. мы пришли к противоречию (ведь не может быть возрастающих подпоследовательностей длиннее $x$).

Таким образом, в самом деле, $y = x$, что и требовалось доказать.

\bf{Восстановление ответа}. Утверждается, что само искомое разбиение на подпоследовательности можно искать жадно, т.е. идя слева направо и относя текущее число в ту подпоследовательность, которая сейчас заканчивается на минимальное число, больше либо равное текущему.



\h2{ Задачи в online judges }

Список задач, которые можно решить по данной тематике:

\ul{

\li \href=http://informatics.mccme.ru/moodle/mod/statements/view3.php?chapterid=1793{MCCME #1793 \bf{"Наибольшая возрастающая подпоследовательность за O(n*log(n))"} ~~~~ [сложность: низкая]}

\li \href=http://community.topcoder.com/stat?c=problem_statement&pm=5922&rd=8075{TopCoder SRM 278 \bf{"500 IntegerSequence"} ~~~~ [сложность: низкая]}

\li \href=http://community.topcoder.com/stat?c=problem_statement&pm=3937&rd=6532{TopCoder SRM 233 \bf{"DIV2 1000 AutoMarket"} ~~~~ [сложность: низкая]}

\li \href=http://codeforces.ru/contest/76/problem/F{Всеукраинская олимпиада школьников по информатике --- задача F \bf{"Турист"} ~~~~ [сложность: средняя]}

\li \href=http://codeforces.ru/problemset/problem/10/D{Codeforces Beta Round #10 --- задача D \bf{"НОВП"} ~~~~ [сложность: средняя]}

\li \href=http://acm.tju.edu.cn/toj/showp2707.html{ACM.TJU.EDU.CN 2707 \bf{"Greatest Common Increasing Subsequence"} ~~~~ [сложность: средняя]}

\li \href=http://www.spoj.pl/problems/SUPPER/{SPOJ #57 \bf{"SUPPER. Supernumbers in a permutation"} ~~~~ [сложность: средняя]}

\li \href=http://community.topcoder.com/stat?c=problem_statement&pm=2967&rd=5881{TopCoder Open 2004 --- Round 4 --- \bf{"1000. BridgeArrangement"} ~~~~ [сложность: высокая]}

}













