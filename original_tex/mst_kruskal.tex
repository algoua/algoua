<h1>Минимальное остовное дерево. Алгоритм Крускала</h1>

<p>Дан взвешенный неориентированный граф. Требуется найти такое поддерево этого графа, которое бы соединяло все его вершины, и при этом обладало наименьшим весом (т.е. суммой весов рёбер) из всех возможных. Такое поддерево называется минимальным остовным деревом или простом минимальным остовом.</p>
<p>Здесь будут рассмотрены несколько важных фактов, связанных с минимальными остовами, затем будет рассмотрен алгоритм Крускала в его простейшей реализации.</p>
<h3>Свойства минимального остова</h3>
<ul>
<li>Минимальный остов <b>уникален, если веса всех рёбер различны</b>. В противном случае, может существовать несколько минимальных остовов (конкретные алгоритмы обычно получают один из возможных остовов).</li>
<li>Минимальный остов является также и <b>остовом с минимальным произведением</b> весов рёбер.<br>(доказывается это легко, достаточно заменить веса всех рёбер на их логарифмы)</li>
<li>Минимальный остов является также и <b>остовом с минимальным весом самого тяжелого ребра</b>.<br>(это утверждение следует из справедливости алгоритма Крускала)</li>
<li><b>Остов максимального веса</b> ищется аналогично остову минимального веса, достаточно поменять знаки всех рёбер на противоположные и выполнить любой из алгоритм минимального остова.</li>
</ul>
<h3>Алгоритм Крускала</h3>
<p>Данный алгоритм был описан Крускалом (Kruskal) в 1956 г.</p>
<p>Алгоритм Крускала изначально помещает каждую вершину в своё дерево, а затем постепенно объединяет эти деревья, объединяя на каждой итерации два некоторых дерева некоторым ребром. Перед началом выполнения алгоритма, все рёбра сортируются по весу (в порядке неубывания). Затем начинается процесс объединения: перебираются все рёбра от первого до последнего (в порядке сортировки), и если у текущего ребра его концы принадлежат разным поддеревьям, то эти поддеревья объединяются, а ребро добавляется к ответу. По окончании перебора всех рёбер все вершины окажутся принадлежащими одному поддереву, и ответ найден.</p>
<h3>Простейшая реализация</h3>
<p>Этот код самым непосредственным образом реализует описанный выше алгоритм, и выполняется за <b>O (M log N + N<sup>2</sup>)</b>. Сортировка рёбер потребует O (M log N) операций. Принадлежность вершины тому или иному поддереву хранится просто с помощью массива tree_id - в нём для каждой вершины хранится номер дерева, которому она принадлежит. Для каждого ребра мы за O (1) определяем, принадлежат ли его концы разным деревьям. Наконец, объединение двух деревьев осуществляется за O (N) простым проходом по массиву tree_id. Учитывая, что всего операций объединения будет N-1, мы и получаем асимптотику <b>O (M log N + N<sup>2</sup>)</b>.</p>
<code>int m;
vector &lt; pair &lt; int, pair&lt;int,int> > > g (m); // вес - вершина 1 - вершина 2

int cost = 0;
vector &lt; pair&lt;int,int> > res;

sort (g.begin(), g.end());
vector&lt;int> tree_id (n);
for (int i=0; i&lt;n; ++i)
	tree_id[i] = i;
for (int i=0; i&lt;m; ++i)
{
	int a = g[i].second.first,  b = g[i].second.second,  l = g[i].first;
	if (tree_id[a] != tree_id[b])
	{
		cost += l;
		res.push_back (make_pair (a, b));
		int old_id = tree_id[b],  new_id = tree_id[a];
		for (int j=0; j&lt;n; ++j)
			if (tree_id[j] == old_id)
				tree_id[j] = new_id;
	}
}</code>
<h3>Улучшенная реализация</h3>
<p>С использованием структуры данных <algohref=dsu>"Система непересекающихся множеств"</algohref> можно написать более быструю реализацию <algohref=mst_kruskal_with_dsu>алгоритма Крускала с асимптотикой O (M log N)</algohref>.</p>