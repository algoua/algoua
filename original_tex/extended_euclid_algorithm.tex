\h1{ Расширенный алгоритм Евклида }


В то время как \algohref=euclid_algorithm{"обычный" алгоритм Евклида} просто находит наибольший общий делитель двух чисел $a$ и $b$, расширенный алгоритм Евклида находит помимо НОД также коэффициенты $x$ и $y$ такие, что:

$$a \cdot x + b \cdot y = {\rm gcd} (a, b).$$

Т.е. он находит коэффициенты, с помощью которых НОД двух чисел выражается через сами эти числа.


\h2{Алгоритм}

Внести вычисление этих коэффициентов в алгоритм Евклида несложно, достаточно вывести формулы, по которым они меняются при переходе от пары $(a,b)$ к паре $(b\%a,a)$ (знаком процента мы обозначаем взятие остатка от деления).

Итак, пусть мы нашли решение $(x_1,y_1)$ задачи для новой пары $(b\%a,a)$:

$$ (b \% a) \cdot x_1 + a \cdot y_1 = g, $$

и хотим получить решение $(x,y)$ для нашей пары $(a,b)$:

$$ a \cdot x + b \cdot y = g. $$

Для этого преобразуем величину $b \% a$:

$$ b \% a = b - \left\lfloor \frac{b}{a} \right\rfloor \cdot a. $$

Подставим это в приведённое выше выражение с $x_1$ и $y_1$ и получим:

$$ g = (b \% a) \cdot x_1 + a \cdot y_1 = \left( b - \left\lfloor \frac{b}{a} \right\rfloor \cdot a \right) \cdot x_1 + a \cdot y_1, $$

и, выполняя перегруппировку слагаемых, получаем:

$$ g = b \cdot x_1 + a \cdot \left( y_1 - \left\lfloor \frac{b}{a} \right\rfloor \cdot x_1 \right). $$

Сравнивая это с исходным выражением над неизвестными $x$ и $y$, получаем требуемые выражения:

$$ \cases{
x = y_1 - \left\lfloor \frac{b}{a} \right\rfloor \cdot x_1, \cr
y = x_1.
} $$


\h2{Реализация}

\code

int gcd (int a, int b, int & x, int & y) {
	if (a == 0) {
		x = 0; y = 1;
		return b;
	}
	int x1, y1;
	int d = gcd (b%a, a, x1, y1);
	x = y1 - (b / a) * x1;
	y = x1;
	return d;
}
\endcode

Это рекурсивная функция, которая по-прежнему возвращает значение НОД от чисел $a$ и $b$, но помимо этого --- также искомые коэффициенты $x$ и $y$ в виде параметров функции, передающихся по ссылкам.

База рекурсии --- случай $a = 0$. Тогда НОД равен $b$, и, очевидно, требуемые коэффициенты $x$ и $y$ равны $0$ и $1$ соответственно. В остальных случаях работает обычное решение, а коэффициенты пересчитываются по вышеописанным формулам.

Расширенный алгоритм Евклида в такой реализации работает корректно даже для отрицательных чисел.


\h2{ Литература }

\ul{
\li \book{Томас Кормен, Чарльз Лейзерсон, Рональд Ривест, Клиффорд Штайн}{Алгоритмы: Построение и анализ}{2005}{cormen.djvu}
}