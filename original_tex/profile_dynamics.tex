<h1>Динамика по профилю. Задача "паркет"</h1>

<p>Типичными задачами на динамику по профилю, являются:</p>
<ul>
<li>найти количество способов замощения поля некоторыми фигурами</li>
<li>найти замощение с наименьшим количеством фигур</li>
<li>найти замощение с минимальным количеством неиспользованных клеток</li>
<li>найти замощение с минимальным количеством фигур такое, что в него нельзя добавить ещё одну фигуру</li>
</ul>
<h2>Задача "Паркет"</h2>
<p>Имеется прямоугольная площадь размером NxM. Нужно найти количество способов замостить эту площадь фигурами 1x2 (пустых клеток не должно оставаться, фигуры не должны накладываться друг на друга).</p>
<p>Построим такую динамику: D[I][Mask], где I=1..N, Mask=0..2^M-1. I обозначает количество строк в текущем поле, а Mask - профиль последней строки в текущем поле. Если j-й бит в Mask равен нулю, то в этом месте профиль проходит на "нормальном уровне", а если 1 - то здесь "выемка" глубиной 1. Ответом, очевидно, будет D[N][0].</p>
<p>Строить такую динамику будем, просто перебирая все I=1..N, все маски Mask=0..2^M-1, и для каждой маски будем делать переходы вперёд, т.е. добавлять к ней новую фигуру всеми возможными способами.</p>
<p><b>Реализация:</b></p>
<code>int n, m;
vector &lt; vector&lt;long long> > d;


void calc (int x = 0, int y = 0, int mask = 0, int next_mask = 0)
{
	if (x == n)
		return;
	if (y >= m)
		d[x+1][next_mask] += d[x][mask];
	else
	{
		int my_mask = 1 &lt;&lt; y;
		if (mask & my_mask)
			calc (x, y+1, mask, next_mask);
		else
		{
			calc (x, y+1, mask, next_mask | my_mask);
			if (y+1 &lt; m && ! (mask & my_mask) && ! (mask & (my_mask &lt;&lt; 1)))
				calc (x, y+2, mask, next_mask);
		}
	}
}


int main()
{
	cin >> n >> m;
	
	d.resize (n+1, vector&lt;long long> (1&lt;&lt;m));
	d[0][0] = 1;
	for (int x=0; x&lt;n; ++x)
		for (int mask=0; mask&lt;(1&lt;&lt;m); ++mask)
			calc (x, 0, mask, 0);

	cout &lt;&lt; d[n][0];

}</code>