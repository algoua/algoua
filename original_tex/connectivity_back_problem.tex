\h1{ Построение графа с указанными величинами вершинной и рёберной связностей и наименьшей из степеней вершин }

Даны величины $\kappa$, $\lambda$, $\delta$ --- это, соответственно, \algohref=vertex_connectivity{вершинная связность}, \algohref=rib_connectivity{рёберная связность} и наименьшая из степеней вершин графа. Требуется построить граф, который бы обладал указанными значениями, или сказать, что такого графа не существует.


\h2{ Соотношение Уитни }

\bf{Соотношение Уитни (Whitney)} (1932 г.) между \algohref=rib_connectivity{рёберной связностью} $\lambda$, \algohref=vertex_connectivity{вершинной связностью} $\kappa$ и наименьшей из степеней вершин $\delta$:

$$ \kappa \le \lambda \le \delta. $$

\bf{Докажем} это утверждение.

Докажем сначала первое неравенство: $\kappa \le \lambda$. Рассмотрим этот набор из $\lambda$ рёбер, делающих граф несвязным. Если мы возьмём от каждого из этих ребёр по одному концу (любому из двух) и удалим из графа, то тем самым с помощью $\le \lambda$ удалённых вершин (поскольку одна и та же вершина могла встретиться дважды) мы сделаем граф несвязным. Таким образом, $\kappa \le \lambda$.

Докажем второе неравенство: $\lambda \le \delta$. Рассмотрим вершину минимальной степени, тогда мы можем удалить все $\delta$ смежных с ней рёбер и тем самым отделить эту вершину от всего остального графа. Следовательно, $\lambda \le \delta$.

Интересно, что неравенство Уитни \bf{нельзя улучшить}: т.е. для любых троек чисел, удовлетворяющих этому неравенству, существует хотя бы один соответствующий граф. Это мы докажем конструктивно, показав, как строятся соответствующие графы.


\h2{ Решение }

Проверим, удовлетворяют ли данные числа $\kappa$, $\lambda$ и $\delta$ соотношению Уитни. Если нет, то ответа не существует.

В противном случае, построим сам граф. Он будет состоять из $2 (\delta + 1)$ вершин, причём первые $\delta + 1$ вершины образуют полносвязный подграф, и вторые $\delta + 1$ вершины также образуют полносвязный подграф. Кроме того, соединим эти две части $\lambda$ рёбрами так, чтобы в первой части эти рёбра были смежны $\lambda$ вершинам, а в другой части --- $\kappa$ вершинам. Легко убедиться в том, что полученный граф будет обладать необходимыми характеристиками.