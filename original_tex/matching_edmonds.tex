\h1{Алгоритм Эдмондса нахождения наибольшего паросочетания в произвольных графах}

Дан неориентированный невзвешенный граф $G$ с $n$ вершинами. Требуется найти в нём наибольшее паросочетание, т.е. такое наибольшее (по мощности) множество $m$ его рёбер, что никакие два ребра из выбранных не инцидентны друг другу (т.е. не имеют общих вершин).

В отличие от случая двудольного графа (см. \algohref=kuhn_matching{Алгоритм Куна}), в графе $G$ могут присутствовать циклы нечётной длины, что значительно усложняет поиск увеличивающих путей.

Приведём сначала теорему Бержа, из которой следует, что, как и в случае двудольных графов, наибольшее паросочетание можно находить при помощи увеличивающих путей.


\h2{Увеличивающие пути. Теорема Бержа}

Пусть зафиксировано некоторое паросочетание $M$. Тогда простая цепь $P = (v_1, v_2, \ldots, v_k)$ называется чередующейся цепью, если в ней рёбра по очереди принадлежат - не принадлежат паросочетанию $M$. Чередующаяся цепь называется увеличивающей, если её первая и последняя вершины не принадлежат паросочетанию. Иными словами, простая цепь $P$ является увеличивающей тогда и только тогда, когда вершина $v_1 \not\in M$, ребро $(v_2,v_3) \in M$, ребро $(v_4,v_5) \in M$, ..., ребро $(v_{k-2},v_{k-1}) \in M$, и вершина $v_k \not\in M$.

\img{edmonds_1.png}

\bf{Теорема Бержа} (Claude Berge, 1957 г.). Паросочетание $M$ является наибольшим тогда и только тогда, когда для него не существует увеличивающей цепи.

\bf{Доказательство необходимости}. Пусть для паросочетания $M$ существует увеличивающая цепь $P$. Покажем, как перейти к паросочетанию большей мощности. Выполним чередование паросочетания $M$ вдоль этой цепи $P$, т.е. включим в паросочетание рёбра $(v_1,v_2)$, $(v_3,v_4)$, ..., $(v_{k-1},v_k)$, и удалим из паросочетания рёбра $(v_2,v_3)$, $(v_4,v_5)$, ..., $(v_{k-2},v_{k-1})$. В результате, очевидно, будет получено корректное паросочетание, мощность которого будет на единицу выше, чем у паросочетания $M$ (т.к. мы добавили $k/2$ рёбер, а удалили $k/2-1$ ребро).

\bf{Доказательство достаточности}. Пусть для паросочетания $M$ не существует увеличивающей цепи, докажем, что оно является наибольшим. Пусть $\overline M$ --- наибольшее паросочетание. Рассмотрим симметрическую разность $\overline G = M \oplus \overline M$ (т.е. множество рёбер, принадлежащих либо $M$, либо $\overline M$, но не обоим одновременно). Покажем, что $\overline G$ содержит одинаковое число рёбер из $M$ и $\overline M$ (т.к. мы исключили из $\overline G$ только общие для них рёбра, то отсюда будет следовать и $|M| = |\overline M|$). Заметим, что $\overline G$ состоит только из простых цепей и циклов (т.к. иначе одной вершине были бы инцидентны сразу два ребра какого-либо паросочетания, что невозможно). Далее, циклы не могут иметь нечётную длину (по той же самой причине). Цепь в $\overline G$ также не может иметь нечётную длину (иначе бы она являлась увеличивающей цепью для $M$, что противоречит условию, или для $\overline M$, что противоречит его максимальности). Наконец, в чётных циклах и цепях чётной длины в $\overline G$ рёбра поочерёдно входят в $M$ и $\overline M$, что и означает, что в $\overline G$ входит одинаковое количество рёбер от $M$ и $\overline M$. Как уже упоминалось выше, отсюда следует, что $|M| = |\overline M|$, т.е. $M$ является наибольшим паросочетанием.

Теорема Бержа даёт основу для алгоритма Эдмондса --- поиск увеличивающих цепей и чередование вдоль них, пока увеличивающие цепи находятся.


\h2{Алгоритм Эдмондса. Сжатие цветков}

Основная проблема заключается в том, как находить увеличивающий путь. Если в графе имеются циклы нечётной длины, то просто запускать обход в глубину/ширину нельзя.

Можно привести простой контрпример, когда при запуске из одной из вершин алгоритм, не обрабатывающий особо циклы нечётной длины (фактически, \algohref=kuhn_matching{Алгоритм Куна}) не найдёт увеличивающий путь, хотя должен. Это цикл длины 3 с висящим на нём ребром, т.е. граф 1-2, 2-3, 3-1, 2-4, и ребро 2-3 взято в паросочетание. Тогда при запуске из вершины 1, если обход пойдёт сначала в вершину 2, то он "упрётся" в вершину 3, вместо того чтобы найти увеличивающую цепь 1-3-2-4. Правда, на этом примере при запуске из вершины 4 алгоритм Куна всё же найдёт эту увеличивающую цепь.

\img{edmonds_2.png}

Тем не менее, можно построить граф, на котором при определённом порядке в списках смежности алгоритм Куна зайдёт в тупик. В качестве примера можно привести такой граф с 6 вершинами и 7 рёбрами: 1-2, 1-6, 2-6, 2-4, 4-3, 1-5, 4-5. Если применить здесь алгоритм Куна, то он найдёт паросочетание 1-6, 2-4, после чего он должен будет обнаружить увеличивающую цепь 5-1-6-2-4-3, однако может так и не обнаружить её (если из вершины 5 он пойдёт сначала в 4, и только потом в 1, а при запуске из вершины 3 он из вершины 2 пойдёт сначала в 1, и только затем в 6).

\img{edmonds_3.png}

Как мы увидели на этом примере, вся проблема в том, что при попадании в цикл нечётной длины обход может пойти по циклу в неправильном направлении. На самом деле, нас интересуют только "насыщенные" циклы, т.е. в которых имеется $k$ насыщенных рёбер, где длина цикла равна $2k+1$. В таком цикле есть ровно одна вершина, не насыщенная рёбрами этого цикла, назовём её \bf{базой} (base). К базовой вершине подходит чередующийся путь чётной (возможно, нулевой) длины, начинающийся в свободной (т.е. не принадлежащей паросочетанию) вершине, и этот путь называется \bf{стеблем} (stem). Наконец, подграф, образованный "насыщенным" нечётным циклом, называется \bf{цветком} (blossom).

\img{edmonds_4.png}

Идея алгоритма Эдмондса (Jack Edmonds, 1965 г.) - в \bf{сжатии цветков} (blossom shrinking). Сжатие цветка --- это сжатие всего нечётного цикла в одну псевдо-вершину (соответственно, все рёбра, инцидентные вершинам этого цикла, становятся инцидентными псевдо-вершине). Алгоритм Эдмондса ищет в графе все цветки, сжимает их, после чего в графе не остаётся "плохих" циклов нечётной длины, и на таком графе (называемом "поверхностным" (surface) графом) уже можно искать увеличивающую цепь простым обходом в глубину/ширину. После нахождения увеличивающей цепи в поверхностном графе необходимо "развернуть" цветки, восстановив тем самым увеличивающую цепь в исходном графе.

Однако неочевидно, что после сжатия цветка не нарушится структура графа, а именно, что если в графе $G$ существовала увеличивающая цепь, то она существует и в графе $\overline G$, полученном после сжатия цветка, и наоборот.

\bf{Теорема Эдмондса}. В графе $\overline G$ существует увеличивающая цепь тогда и только тогда, когда существует увеличивающая цепь в $G$.

\bf{Доказательство}. Итак, пусть граф $\overline G$ был получен из графа $G$ сжатием одного цветка (обозначим через $B$ цикл цветка, и через $\overline B$ соответствующую сжатую вершину), докажем утверждение теоремы. Вначале заметим, что достаточно рассматривать случай, когда база цветка является свободной вершиной (не принадлежащей паросочетанию). Действительно, в противном случае в базе цветка оканчивается чередующийся путь чётной длины, начинающийся в свободной вершине. Прочередовав паросочетание вдоль этого пути, мощность паросочетания не изменится, а база цветка станет свободной вершиной. Итак, при доказательстве можно считать, что база цветка является свободной вершиной.

\bf{Доказательство необходимости}. Пусть путь $P$ является увеличивающим в графе $G$. Если он не проходит через $B$, то тогда, очевидно, он будет увеличивающим и в графе $\overline G$. Пусть $P$ проходит через $B$. Тогда можно не теряя общности считать, что путь $P$ представляет собой некоторый путь $P_1$, не проходящий по вершинам $B$, плюс некоторый путь $P_2$, проходящий по вершинам $B$ и, возможно, другим вершинам. Но тогда путь $P_1 + \overline B$ будет являться увеличивающим путём в графе $\overline G$, что и требовалось доказать.

\bf{Доказательство достаточности}. Пусть путь $\overline P$ является увеличивающим путём в графе $\overline G$. Снова, если путь $\overline P$ не проходит через $\overline B$, то путь $\overline P$ без изменений является увеличивающим путём в $G$, поэтому этот случай мы рассматривать не будем.

Рассмотрим отдельно случай, когда $\overline P$ начинается со сжатого цветка $\overline B$, т.е. имеет вид $(\overline B, c, \ldots)$. Тогда в цветке $B$ найдётся соответствующая вершина $v$, которая связана (ненасыщенным) ребром с $c$. Осталось только заметить, что из базы цветка всегда найдётся чередующийся путь чётной длины до вершины $v$. Учитывая всё вышесказанное, получаем, что путь $P = (b, \ldots, v, c, ...)$ является увеличивающим путём в графе $G$.

Пусть теперь путь $\overline P$ проходит через псевдо-вершину $\overline B$, но не начинается и не заканчивается в ней. Тогда в $\overline P$ есть два ребра, проходящих через $\overline B$, пусть это $(a, \overline B)$ и $(\overline B, c)$. Одно из них обязательно должно принадлежать паросочетанию $M$, однако, т.к. база цветка не насыщена, а все остальные вершины цикла цветка $B$ насыщены рёбрами цикла, то мы приходим к противоречию. Таким образом, этот случай просто невозможен.

Итак, мы рассмотрели все случаи и в каждом из них показали справедливость теоремы Эдмондса.

\bf{Общая схема алгоритма Эдмондса} принимает следующий вид:

\code
void edmonds() {
	for (int i=0; i<n; ++i)
		if (вершина i не в паросочетании) {
			int last_v = find_augment_path (i);
			if (last_v != -1)
				выполнить чередование вдоль пути из i в last_v;
		}
}

int find_augment_path (int root) {
	обход в ширину:
		int v = текущая_вершина;
		перебрать все рёбра из v
			если обнаружили цикл нечётной длины, сжать его
			если пришли в свободную вершину, return
			если пришли в несвободную вершину, то добавить
				в очередь смежную ей в паросочетании
	return -1;
}
\endcode


\h2{Эффективная реализация}

Сразу оценим асимптотику. Всего имеется $n$ итераций, на каждой из которых выполняется обход в ширину за $O(m)$, кроме того, могут происходить операции сжатия цветков --- их может быть $O(n)$. Таким образом, если мы научимся сжимать цветок за $O(n)$, то общая асимптотика алгоритма составит $O(n \cdot (m + n^2)) = O(n^3)$.

Основную сложность представляют операции сжатия цветков. Если выполнять их, непосредственно объединяя списки смежности в один и удаляя из графа лишние вершины, то асимптотика сжатия одного цветка будет $O(m)$, кроме того, возникнут сложности при "разворачивании" цветков.

Вместо этого будем для каждой вершины графа $G$ поддерживать указатель на базу цветка, которому она принадлежит (или на себя, если вершина не принадлежит никакому цветку). Нам надо решить две задачи: сжатие цветка за $O(n)$ при его обнаружении, а также удобное сохранение всей информации для последующего чередования вдоль увеличивающего пути.

Итак, одна итерация алгоритма Эдмондса представляет собой обход в ширину, выполняемый из заданной свободной вершины $\rm root$. Постепенно будет строиться дерево обхода в ширину, причём путь в нём до любой вершины будет являться чередующимся путём, начинающимся со свободной вершины $\rm root$. Для удобства программирования будем класть в очередь только те вершины, расстояние до которых в дереве путей чётно (будем называть такие вершины чётными --- т.е. это корень дерева, и вторые концы рёбер в паросочетании). Само дерево будем хранить в виде массива предков $\rm p[]$, в котором для каждой нечётной вершины (т.е. до которой расстояние в дереве путей нечётно, т.е. это первые концы рёбер в паросочетании) будем хранить предка - чётную вершину. Таким образом, для восстановления пути из дерева нам надо поочерёдно пользоваться массивами $\rm p[]$ и $\rm match[]$, где $\rm match[]$ --- для каждой вершины содержит смежную ей в паросочетании, или $-1$, если таковой нет.

Теперь становится понятно, как обнаруживать циклы нечётной длины. Если мы из текущей вершины $v$ в процессе обхода в ширину приходим в такую вершину $u$, являющуюся корнем $\rm root$ или принадлежащую паросочетанию и дереву путей (т.е. $\rm p[match[]]$ от которой не равно -1), то мы обнаружили цветок. Действительно, при выполнении этих условий и вершина $v$, и вершина $u$ являются чётными вершинами. Расстояние от них до их наименьшего общего предка имеет одну чётность, поэтому найденный нами цикл имеет нечётную длину.

Научимся \bf{сжимать цикл}. Итак, мы обнаружили нечётный цикл при рассмотрении ребра $(v,u)$, где $u$ и $v$ --- чётные вершины. Найдём их наименьшего общего предка $b$, он и будет базой цветка. Нетрудно заметить, что база тоже является чётной вершиной (поскольку у нечётных вершин в дереве путей есть только один сын). Однако надо заметить, что $b$ --- это, возможно, псевдовершина, поэтому мы фактически найдём базу цветка, являющегося наименьшим общим предком вершин $v$ и $u$. Реализуем сразу нахождение наименьшего общего предка (нас вполне устраивает асимптотика $O(n)$):

\code
int lca (int a, int b) {
	bool used[MAXN] = { 0 };
	// поднимаемся от вершины a до корня, помечая все чётные вершины
	for (;;) {
		a = base[a];
		used[a] = true;
		if (match[a] == -1)  break; // дошли до корня
		a = p[match[a]];
	}
	// поднимаемся от вершины b, пока не найдём помеченную вершину
	for (;;) {
		b = base[b];
		if (used[b])  return b;
		b = p[match[b]];
	}
}
\endcode

Теперь нам надо выявить сам цикл --- пройтись от вершин $v$ и $u$ до базы $b$ цветка. Будет более удобно, если мы пока просто пометим в каком-то специальном массиве (назовём его $\rm blossom[]$) вершины, принадлежащие текущему цветку. После этого нам надо будет симитировать обход в ширину из псевдо-вершины --- для этого достаточно положить в очередь обхода в ширину все вершины, лежащие на цикле цветка. Тем самым мы избежим явного объединения списков смежности.

Однако остаётся ещё одна проблема: корректное восстановление путей по окончании обхода в ширину. Для него мы сохраняли массив предков $\rm p[]$. Но после сжатия цветков возникает единственная проблема: обход в ширину продолжился сразу из всех вершин цикла, в том числе и нечётных, а массив предков у нас предназначался для восстановления путей по чётным вершинам. Более того, когда в сжатом графе найдётся увеличивающая цепь, проходящая через цветок, она вообще будет проходить по этому циклу в таком направлении, что в дереве путей это будет представляться движением вниз. Однако все эти проблемы изящно решаются таким манёвром: при сжатии цикла, проставим предков для всех его чётных вершин (кроме базы), чтобы эти "предки" указывали на соседнюю вершину в цикле. Для вершин $u$ и $v$, если они также не базы, направим указатели предков друг на друга. В результате, если при восстановлении увеличивающего пути мы придём в цикл цветка в нечётную вершину, путь по предкам будет восстановлен корректно, и приведёт в базу цветка (из которой он уже дальше будет восстанавливаться нормально).

\img{edmonds_5.png}

Итак, мы готовы реализовать сжатие цветка:

\code
int v, u; // ребро (v,u), при рассмотрении которого был обнаружен цветок
int b = lca (v, u);
memset (blossom, 0, sizeof blossom);
mark_path (v, b, u);
mark_path (u, b, v);
\endcode

где функция $\rm mark\_path()$ проходит по пути от вершины до базы цветка, проставляет в специальном массиве $\rm blossom[]$ для них $\rm true$ и проставляет предков для чётных вершин. Параметр $\rm children$ --- сын для самой вершины $v$ (с помощью этого параметра мы замкнём цикл в предках).

\code
void mark_path (int v, int b, int children) {
	while (base[v] != b) {
		blossom[base[v]] = blossom[base[match[v]]] = true;
		p[v] = children;
		children = match[v];
		v = p[match[v]];
	}
}
\endcode

Наконец, реализуем основную функцию --- $\rm find\_path ~ (int ~ root)$, которая будет искать увеличивающий путь из свободной вершины $\rm root$ и возвращать последнюю вершину этого пути, либо $-1$, если увеличивающий путь не найден.

Вначале произведём инициализацию:

\code
int find_path (int root) {
	memset (used, 0, sizeof used);
	memset (p, -1, sizeof p);
	for (int i=0; i<n; ++i)
		base[i] = i;
\endcode

Далее идёт обход в ширину. Рассматривая очередное ребро $(v, to)$, у нас есть несколько вариантов:

\ul{

\li Ребро несуществующее. Под этим мы подразумеваем, что $v$ и $to$ принадлежат одной сжатой псевдо-вершине (${\rm base}[v] == {\rm base}[to]$), поэтому в текущем поверхностном графе этого ребра нет. Кроме этого случая, есть ещё один случай: когда ребро $(v, to)$ уже принадлежит текущему паросочетанию; т.к. мы считаем, что вершина $v$ является чётной вершиной, то проход по этому ребру означает в дереве путей подъём к предку вершины $v$, что недопустимо.

\code
if (base[v] == base[to] || match[v] == to)  continue;
\endcode

\li Ребро замыкает цикл нечётной длины, т.е. обнаруживается цветок. Как уже упоминалось выше, цикл нечётной длины обнаруживается при выполнении условия:

\code
if (to == root || match[to] != -1 && p[match[to]] != -1)
\endcode

В этом случае нужно выполнить сжатие цветка. Выше уже подробно разбирался этот процесс, здесь приведём его реализацию:

\code
int curbase = lca (v, to);
memset (blossom, 0, sizeof blossom);
mark_path (v, curbase, to);
mark_path (to, curbase, v);
for (int i=0; i<n; ++i)
	if (blossom[base[i]]) {
		base[i] = curbase;
		if (!used[i]) {
			used[i] = true;
			q[qt++] = i;
		}
	}
\endcode

\li Иначе --- это "обычное" ребро, поступаем как и в обычном поиске в ширину. Единственная тонкость --- при проверке, что эту вершину мы ещё не посещали, надо смотреть не в массив $\rm used$, а в массив $p$ --- именно он заполняется для посещённых нечётных вершин. Если мы в вершину $to$ ещё не заходили, и она оказалась ненасыщенной, то мы нашли увеличивающую цепь, заканчивающуюся на вершине $to$, возвращаем её.

\code
if (p[to] == -1) {
	p[to] = v;
	if (match[to] == -1)
		return to;
	to = match[to];
	used[to] = true;
	q[qt++] = to;
}
\endcode

}

Итак, полная реализация функции $\rm find\_path()$:

\code
int find_path (int root) {
	memset (used, 0, sizeof used);
	memset (p, -1, sizeof p);
	for (int i=0; i<n; ++i)
		base[i] = i;

	used[root] = true;
	int qh=0, qt=0;
	q[qt++] = root;
	while (qh < qt) {
		int v = q[qh++];
		for (size_t i=0; i<g[v].size(); ++i) {
			int to = g[v][i];
			if (base[v] == base[to] || match[v] == to)  continue;
			if (to == root || match[to] != -1 && p[match[to]] != -1) {
				int curbase = lca (v, to);
				memset (blossom, 0, sizeof blossom);
				mark_path (v, curbase, to);
				mark_path (to, curbase, v);
				for (int i=0; i<n; ++i)
					if (blossom[base[i]]) {
						base[i] = curbase;
						if (!used[i]) {
							used[i] = true;
							q[qt++] = i;
						}
					}
			}
			else if (p[to] == -1) {
				p[to] = v;
				if (match[to] == -1)
					return to;
				to = match[to];
				used[to] = true;
				q[qt++] = to;
			}
		}
	}
	return -1;
}
\endcode

Наконец, приведём определения всех глобальных массивов, и реализацию основной программы нахождения наибольшего паросочетания:

\code
const int MAXN = ...; // максимально возможное число вершин во входном графе

int n;
vector<int> g[MAXN];
int match[MAXN], p[MAXN], base[MAXN], q[MAXN];
bool used[MAXN], blossom[MAXN];

...

int main() {
	... чтение графа ...

	memset (match, -1, sizeof match);
	for (int i=0; i<n; ++i)
		if (match[i] == -1) {
			int v = find_path (i);
			while (v != -1) {
				int pv = p[v],  ppv = match[pv];
				match[v] = pv,  match[pv] = v;
				v = ppv;
			}
		}
}
\endcode


\h2{Оптимизация: предварительное построение паросочетания}

Как и в случае \algohref=kuhn_matching{Алгоритма Куна}, перед выполнением алгоритма Эдмондса можно каким-нибудь простым алгоритмом построить предварительное паросочетание. Например, таким жадным алгоритмом:
\code
for (int i=0; i<n; ++i)
	if (match[i] == -1)
		for (size_t j=0; j<g[i].size(); ++j)
			if (match[g[i][j]] == -1) {
				match[g[i][j]] = i;
				match[i] = g[i][j];
				break;
			}
\endcode

Такая оптимизация значительно (до нескольких раз) ускорит работу алгоритма на случайных графах.


\h2{Случай двудольного графа}

В двудольных графах отсутствуют циклы нечётной длины, и, следовательно, код, выполняющий сжатие цветков, никогда не выполнится. Удалив мысленно все части кода, обрабатывающие сжатие цветков, мы получим \algohref=kuhn_matching{Алгоритм Куна} практически в чистом виде. Таким образом, на двудольных графах алгоритм Эдмондса вырождается в \algohref=kuhn_matching{алгоритм Куна} и работает за $O (n m)$.


\h2{Дальнейшая оптимизация}

Во всех вышеописанных операциях с цветками легко угадываются операции с непересекающимися множествами, которые можно выполнять намного эффективнее (см. \algohref=dsu{Система непересекающихся множеств}). Если переписать алгоритм с использованием этой структуры, то асимптотика алгоритма понизится до $O (n m)$. Таким образом, для произвольных графов мы получили ту же асимптотическую оценку, что и в случае двудольных графов (алгоритм Куна), но заметно более сложным алгоритмом.
