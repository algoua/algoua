\h1{Покрытие путями ориентированного ациклического графа}

Дан ориентированный ациклический граф $G$. Требуется покрыть его наименьшим числом путей, т.е. найти наименьшее по мощности множество непересекающихся по вершинам простых путей, таких, что каждая вершина принадлежит какому-либо пути.


\h2{Сведение к двудольному графу}

Пусть дан граф $G$ с $n$ вершинами. Построим соответствующий ему двудольный граф $H$ стандартным образом, т.е.: в каждой доле графа $H$ будет по $n$ вершин, обозначим их через $a_i$ и $b_i$ соответственно. Тогда для каждого ребра $(i, j)$ исходного графа $G$ проведём соответствующее ребро $(a_i, b_j)$.

Каждому ребру $G$ соответствует одно ребро $H$, и наоборот. Если мы рассмотрим в $G$ любой путь $P = (v_1, v_2, \ldots, v_k)$, то ему ставится в соответствие набор рёбер $(a_{v_1}, b_{v_2}), (a_{v_2}, b_{v_3}), ..., (a_{v_{k-1}}, b_{v_k}) $.

Более просто для понимания будет, если мы добавим "обратные" рёбра, т.е. образуем граф $\overline H$ из графа $H$ добавлением рёбер вида $(b_i, a_i), i=1 \ldots N$. Тогда пути $P = (v_1, v_2, \ldots, v_k)$ в графе $\overline H$ будет соответствовать путь $\overline Q = (a_{v_1}, b_{v_2}, a_{v_2}, b_{v_3}, ..., a_{v_{k-1}}, b_{v_k})$.

Обратно, рассмотрим любой путь $\overline Q$ в графе $\overline H$, начинающийся в первой доле и заканчивающийся во второй доле. Очевидно, $\overline Q$ снова будет иметь вид $\overline Q = (a_{v_1}, b_{v_2}, a_{v_2}, b_{v_3}, ..., a_{v_{k-1}}, b_{v_k})$, и ему можно поставить в соответствие в графе $G$ путь $P = (v_1, v_2, \ldots, v_k)$. Однако здесь есть одна тонкость: $v_1$ могло совпадать с $v_k$, поэтому путь $P$ получился бы циклом. Однако по условию граф $G$ ациклический, поэтому это вообще невозможно (это единственное место, где используется ацикличность графа $G$; тем не менее, на циклические графы описываемый здесь метод вообще нельзя обобщить).

Итак, всякому простому пути в графе $\overline H$, начинающемуся в первой доле и заканчивающемуся во второй, можно поставить в соответствие простой путь в графе $G$, и наоборот. Но заметим, что такой путь в графе $\overline H$ --- это \bf{паросочетание} в графе $H$. Таким образом, любому пути из $G$ можно поставить в соответствие паросочетание в графе $H$, и наоборот. Более того, непересекающимся путям в $G$ соответствуют непересекающиеся паросочетания в $H$.

Последний шаг. Заметим, что чем больше путей есть в нашем наборе, тем меньше все эти пути содержат рёбер. А именно, если есть $p$ непересекающихся путей, покрывающих все $n$ вершин графа, то они вместе содержат $r = n - p$ рёбер. Итак, чтобы минимизировать число путей, мы должны \bf{максимизировать число рёбер} в них.

Итак, мы свели задачу к нахождению максимального паросочетания в двудольном графе $H$. После нахождения этого паросочетания (см. \algohref=kuhn_matching{Алгоритм Куна}) мы должны преобразовать его в набор путей в $G$ (это делается тривиальным алгоритмом, неоднозначностей здесь не возникает). Некоторые вершины могут остаться ненасыщенными паросочетанием, в таком случае в ответ надо добавить пути нулевой длины из каждой из этих вершин.


\h2{Взвешенный случай}

Взвешенный случай не сильно отличается от невзвешенного, просто в графе $H$ на рёбрах появляются веса, и требуется найти уже паросочетание наименьшего веса. Восстанавливая ответ аналогично невзвешенному случаю, мы получим покрытие графа наименьшим числом путей, а при равенстве --- наименьшим по стоимости.