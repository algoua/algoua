\h1{ Суффиксный массив }

Дана строка $s[0 \ldots n-1]$ длины $n$.

$i$-ым \bf{суффиксом} строки называется подстрока $s[i \ldots n-1]$, $i=0 \ldots n-1$.

Тогда \bf{суффиксным массивом} строки $s$ называется перестановка индексов суффиксов $p[0 \ldots n-1]$, $p[i] \in [0;n-1]$, которая задаёт порядок суффиксов в порядке лексикографической сортировки. Иными словами, нужно выполнить сортировку всех суффиксов заданной строки.

Например, для строки $s=abaab$ суффиксный массив будет равен:

$$ (2,3,0,4,1) $$


\h2{ Построение за $O (n \log n)$ }

Строго говоря, описываемый ниже алгоритм будет выполнять сортировку не суффиксов, а \bf{циклических сдвигов} строки. Однако из этого алгоритма легко получить и алгоритм сортировки суффиксов: достаточно приписать в конец строки произвольный символ, который заведомо меньше любого символа, из которого может состоять строка (например, это может быть доллар или шарп; в языке C в этих целях можно использовать уже имеющийся нулевой символ).

Сразу заметим, что поскольку мы сортируем циклические сдвиги, то и подстроки мы будем рассматривать \bf{циклические}: под подстрокой $s[i \ldots j]$, когда $i > j$, понимается подстрока $s[i \ldots n-1] + s[0 \ldots j]$. Кроме того, предварительно все индексы берутся по модулю длины строки (в целях упрощения формул я буду опускать явные взятия индексов по модулю).

Рассматриваемый нами алгоритм состоит из примерно $\log n$ фаз. На $k$-ой фазе ($k = 0 \ldots \lceil \log n \rceil$) сортируются циклические подстроки длины $2^k$. На последней, $\lceil \log n \rceil$-ой фазе, будут сортироваться подстроки длины $2^{\lceil \log n \rceil} > n$, что эквивалентно сортировке циклических сдвигов.

На каждой фазе алгоритм помимо перестановки $p[0 \ldots n-1]$ индексов циклических подстрок будет поддерживать для каждой циклической подстроки, начинающейся в позиции $i$ с длиной $2^k$, \bf{номер $c[i]$ класса эквивалентности}, которому эта подстрока принадлежит. В самом деле, среди подстрок могут быть одинаковые, и алгоритму понадобится информация об этом. Кроме того, номера $c[i]$ классов эквивалентности будем давать таким образом, чтобы они сохраняли и информацию о порядке: если один суффикс меньше другого, то и номер класса он должен получить меньший. Классы будем для удобства нумеровать с нуля. Количество классов эквивалентности будем хранить в переменной $\rm classes$.

Приведём \bf{пример}. Рассмотрим строку $s=aaba$. Значения массивов $p[]$ и $c[]$ на каждой стадии с нулевой по вторую таковы:

$$ \matrix{
0: & p=(0,1,3,2) & c=(0,0,1,0) \cr
1: & p=(0,3,1,2) & c=(0,1,2,0) \cr
2: & p=(3,0,1,2) & c=(1,2,3,0) \cr
} $$

Стоит отметить, что в массиве $p[]$ возможны неоднозначности. Например, на нулевой фазе массив мог равняться: $p=(3,1,0,2)$. То, какой именно вариант получится, зависит от конкретной реализации алгоритма, но все варианты одинаково правильны. В то же время, в массиве $c[]$ никаких неоднозначностей быть не могло.

Перейдём теперь к построению \bf{алгоритма}. Входные данные:

\code
char *s; // входная строка
int n; // длина строки

// константы
const int maxlen = ...; // максимальная длина строки
const int alphabet = 256; // размер алфавита, <= maxlen
\endcode

На \bf{нулевой фазе} мы должны отсортировать циклические подстроки длины $1$, т.е. отдельные символы строки, и разделить их на классы эквивалентности (просто одинаковые символы должны быть отнесены к одному классу эквивалентности). Это можно сделать тривиально, например, сортировкой подсчётом. Для каждого символа посчитаем, сколько раз он встретился. Потом по этой информации восстановим массив $p[]$. После этого, проходом по массиву $p[]$ и сравнением символов, строится массив $c[]$.

\code
int p[maxlen], cnt[maxlen], c[maxlen];
memset (cnt, 0, alphabet * sizeof(int));
for (int i=0; i<n; ++i)
	++cnt[s[i]];
for (int i=1; i<alphabet; ++i)
	cnt[i] += cnt[i-1];
for (int i=0; i<n; ++i)
	p[--cnt[s[i]]] = i;
c[p[0]] = 0;
int classes = 1;
for (int i=1; i<n; ++i) {
	if (s[p[i]] != s[p[i-1]])  ++classes;
	c[p[i]] = classes-1;
}
\endcode

Далее, пусть мы выполнили $k-1$-ю фазу (т.е. вычислили значения массивов $p[]$ и $c[]$ для неё), теперь научимся за $O(n)$ выполнять \bf{следующую, $k$-ю, фазу}. Поскольку фаз всего $O(\log n)$, это даст нам требуемый алгоритм с временем $O(n \log n)$.

Для этого заметим, что циклическая подстрока длины $2^k$ состоит из двух подстрок длины $2^{k-1}$, которые мы можем сравнивать между собой за $O(1)$, используя информацию с предыдущей фазы --- номера $c[]$ классов эквивалентности. Таким образом, для подстроки длины $2^k$, начинающейся в позиции $i$, вся необходимая информация содержится в паре чисел $(c[i], c[i+2^{k-1}])$ (повторимся, мы используем массив $c[]$ с предыдущей фазы).

$$ \ldots \overbrace{ \underbrace{ s_i \ldots s_{i+2^{k-1}-1} }_{{\rm length}=2^{k-1},{\rm class}=c[i]}\ \ \underbrace{ s_{i+2^{k-1}} \ldots s_{i+2^k-1} }_{{\rm length}=2^{k-1},\ {\rm class}=c[i+2^{k-1}]} }^{{\rm length}=2^k} \ldots \overbrace{ \underbrace{ s_j \ldots s_{j+2^{k-1}-1} }_{{\rm length}=2^{k-1},{\rm class}=c[j]}\ \ \underbrace{ s_{j+2^{k-1}} \ldots s_{j+2^k-1} }_{{\rm length}=2^{k-1},{\rm class}=c[j+2^{k-1}]} }^{{\rm length}=2^k} \ldots $$

Это даёт нам весьма простое решение: \bf{отсортировать} подстроки длины $2^k$ просто \bf{по} этим \bf{парам чисел}, это и даст нам требуемый порядок, т.е. массив $p[]$. Однако обычная сортировка, выполняющаяся за время $O(n \log n)$, нас не устроит --- это даст алгоритм построения суффиксного массива с временем $O(n \log^2 n)$ (зато этот алгоритм несколько проще в написании, чем описываемый ниже).

Как быстро выполнить такую сортировку пар? Поскольку элементы пар не превосходят $n$, то можно выполнить сортировку подсчётом. Однако для достижения лучшей скрытой в асимптотике константы вместо сортировки пар придём к сортировке просто чисел.

Воспользуемся здесь приёмом, на котором основана так называемая \bf{цифровая сортировка}: чтобы отсортировать пары, отсортируем их сначала по вторым элементам, а затем --- по первым элементам (но уже обязательно стабильной сортировкой, т.е. не нарушающей относительного порядка элементов при равенстве). Однако отдельно вторые элементы уже упорядочены --- этот порядок задан в массиве $p[]$ от предыдущей фазы. Тогда, чтобы упорядочить пары по вторым элементам, надо просто от каждого элемента массива $p[]$ отнять $2^{k-1}$ --- это даст нам порядок сортировки пар по вторым элементам (ведь $p[]$ даёт упорядочение подстрок длины $2^{k-1}$, и при переходе к строке вдвое большей длины эти подстроки становятся их вторыми половинками, поэтому от позиции второй половинки отнимается длина первой половинки).

Таким образом, с помощью всего лишь вычитаний от элементов массива $p[]$ мы производим сортировку по вторым элементам пар. Теперь надо произвести стабильную сортировку по первым элементам пар, её уже можно выполнить за $O(n)$ с помощью сортировки подсчётом.

Осталось только пересчитать номера $c[]$ классов эквивалентности, но их уже легко получить, просто пройдя по полученной новой перестановке $p[]$ и сравнивая соседние элементы (опять же, сравнивая как пары двух чисел).

Приведём \bf{реализацию} выполнения всех фаз алгоритма, кроме нулевой. Вводятся дополнительно временные массивы $pn$ и $cn$ ($pn$ --- содержит перестановку в порядке сортировки по вторым элементам пар, $cn$ --- новые номера классов эквивалентности).

\code
int pn[maxlen], cn[maxlen];
for (int h=0; (1<<h)<n; ++h) {
	for (int i=0; i<n; ++i) {
		pn[i] = p[i] - (1<<h);
		if (pn[i] < 0)  pn[i] += n;
	}
	memset (cnt, 0, classes * sizeof(int));
	for (int i=0; i<n; ++i)
		++cnt[c[pn[i]]];
	for (int i=1; i<classes; ++i)
		cnt[i] += cnt[i-1];
	for (int i=n-1; i>=0; --i)
		p[--cnt[c[pn[i]]]] = pn[i];
	cn[p[0]] = 0;
	classes = 1;
	for (int i=1; i<n; ++i) {
		int mid1 = (p[i] + (1<<h)) % n,  mid2 = (p[i-1] + (1<<h)) % n;
		if (c[p[i]] != c[p[i-1]] || c[mid1] != c[mid2])
			++classes;
		cn[p[i]] = classes-1;
	}
	memcpy (c, cn, n * sizeof(int));
}
\endcode

Этот алгоритм требует $O(n \log n)$ времени и $O(n)$ памяти. Впрочем, если учитывать ещё размер $k$ алфавита, то время работы становится $O((n+k) \log n)$, а размер памяти --- $O(n+k)$.


\h2{ Применения }


\h3{ Нахождение наименьшего циклического сдвига строки }

Вышеописанный алгоритм производит сортировку циклических сдвигов (если к строке не приписывать доллар), а потому $p[0]$ даст искомую позицию наименьшего циклического сдвига. Время работы --- $O(n \log n)$.


\h3{ Поиск подстроки в строке }

Пусть требуется в тексте $t$ искать строку $s$ в режиме онлайн (т.е. заранее строку $s$ нужно считать неизвестной). Построим суффиксный массив для текста $t$ за $O (|t| \log |t|)$. Теперь подстроку $s$ будем искать следующим образом: заметим, что искомое вхождение должно быть префиксом какого-либо суффикса $t$. Поскольку суффиксы у нас упорядочены (это даёт нам суффиксный массив), то подстроку $s$ можно искать бинарным поиском по суффиксам строки. Сравнение текущего суффикса и подстроки $s$ внутри бинарного поиска можно производить тривиально, за $O(|p|)$. Тогда асимптотика поиска подстроки в тексте становится $O(|p| \log |t|)$.


\h3{ Сравнение двух подстрок строки }

Требуется по заданной строке $s$, произведя некоторый её препроцессинг, научиться за $O(1)$ отвечать на запросы сравнения двух произвольных подстрок (т.е. проверка, что первая подстрока равна/меньше/больше второй).

Построим суффиксный массив за $O (|s| \log |s|)$, при этом сохраним промежуточные результаты: нам понадобятся массивы $c[]$ от каждой фазы. Поэтому памяти потребуется тоже $O (|s| \log |s|)$.

Используя эту информацию, мы можем за $O(1)$ сравнивать любые две подстроки длины, равной степени двойки: для этого достаточно сравнить номера классов эквивалентности из соответствующей фазы. Теперь надо обобщить этот способ на подстроки произвольной длины.

Пусть теперь поступил очередной запрос сравнения двух подстрок длины $l$ с началами в индексах $i$ и $j$. Найдём наибольшую длину блока, помещающегося внутри подстроки такой длины, т.е. наибольшее $k$ такое, что $2^k \le l$. Тогда сравнение двух подстрок можно заменить сравнением двух пар перекрывающихся блоков длины $2^k$: сначала надо сравнить два блока, начинающихся в позициях $i$ и $j$, а при равенстве --- сравнить два блока, заканчивающихся в позициях $i+l-1$ и $j+l-1$:

$$ \ldots \overbrace{ \underbrace{ s_i \ldots s_{i+l-2^k} \ldots s_{i+2^k-1} }_{2^k} \ldots s_{i+l-1} }^{\rm first} \ldots \overbrace{ \underbrace{ s_j \ldots s_{j+l-2^k} \ldots s_{j+2^k-1} }_{2^k} \ldots s_{j+l-1} }^{\rm second} \ldots $$
$$ \ldots \overbrace{ s_i \ldots \underbrace{ s_{i+l-2^k} \ldots s_{i+2^k-1} \ldots s_{i+l-1} }_{2^k} }^{\rm first} \ldots \overbrace{ s_j \ldots \underbrace{ s_{j+l-2^k} \ldots s_{j+2^k-1} \ldots s_{j+l-1} }_{2^k} }^{\rm second} \ldots $$

Таким образом, реализация получается примерно такой (здесь считается, что вызывающая процедура сама вычисляет $k$, поскольку сделать это за константное время не так легко (по-видимому, быстрее всего --- предпосчётом), но в любом случае это не имеет отношения к применению суффиксного массива):

\code
int compare (int i, int j, int l, int k) {
	pair<int,int> a = make_pair (c[k][i], c[k][i+l-(1<<k)]);
	pair<int,int> b = make_pair (c[k][j], c[k][j+l-(1<<k)]);
	return a == b ? 0 : a < b ? -1 : 1;
}
\endcode


\h3{ Наибольший общий префикс двух подстрок: способ с дополнительной памятью }

Требуется по заданной строке $s$, произведя некоторый её препроцессинг, научиться за $O(\log |s|)$ отвечать на запросы наибольшего общего префикса (longest common prefix, lcp) для двух произвольных суффиксов с позициями $i$ и $j$.

Способ, описываемый здесь, требует $O (|s| \log |s|)$ дополнительной памяти; другой способ, использующий линейный объём памяти, но неконстантное время ответа на запрос, описан в следующем разделе.

Построим суффиксный массив за $O (|s| \log |s|)$, при этом сохраним промежуточные результаты: нам понадобятся массивы $c[]$ от каждой фазы. Поэтому памяти потребуется тоже $O (|s| \log |s|)$.

Пусть теперь поступил очередной запрос: пара индексов $i$ и $j$. Воспользуемся тем, что мы можем за $O(1)$ сравнивать любые две подстроки длины, являющейся степенью двойки. Для этого будем перебирать степень двойки (от большей к меньшей), и для текущей степени проверять: если подстроки такой длины совпадают, то к ответу прибавить эту степень двойки, а наибольший общий префикс продолжим искать справа от одинаковой части, т.е. к $i$ и $j$ надо прибавить текущую степень двойки.

Реализация:

\code
int lcp (int i, int j) {
	int ans = 0;
	for (int k=log_n; k>=0; --k)
		if (c[k][i] == c[k][j]) {
			ans += 1<<k;
			i += 1<<k;
			j += 1<<k;
		}
	return ans;
}
\endcode

Здесь через $\rm log\_n$ обозначена константа, равная логарифму $n$ по основанию 2, округлённому вниз.


\h3{ Наибольший общий префикс двух подстрок: способ без дополнительной памяти. Наибольший общий префикс двух соседних суффиксов }

Требуется по заданной строке $s$, произведя некоторый её препроцессинг, научиться отвечать на запросы наибольшего общего префикса (longest common prefix, lcp) для двух произвольных суффиксов с позициями $i$ и $j$.

В отличие от предыдущего метода, описываемый здесь будет выполнять препроцессинг строки за $O(n \log n)$ времени с $O(n)$ памяти. Результатом этого препроцессинга будет являться массив (который сам по себе является важным источником информации о строке, и потому использоваться для решения других задач). Ответы же на запрос будут производиться как результат выполнения запроса RMQ (минимум на отрезке, range minimum query) в этом массиве, поэтому при разных реализациях можно получить как логарифмическое, так и константное времена работы.

Базой для этого алгоритма является следующая идея: найдём каким-нибудь образом наибольшие общие префиксы для каждой \bf{соседней в порядке сортировки пары суффиксов}. Иными словами, построим массив ${\rm lcp}[0 \ldots n-2]$, где ${\rm lcp}[i]$ равен наибольшему общему префиксу суффиксов $p[i]$ и $p[i+1]$. Этот массив даст нам ответ для любых двух соседних суффиксов строки. Тогда ответ для любых двух суффиксов, не обязательно соседних, можно получить по этому массиву. В самом деле, пусть поступил запрос с некоторыми номерами суффиксов $i$ и $j$. Найдём эти индексы в суффиксном массиве, т.е. пусть $k_1$ и $k_2$ --- их позиции в массиве $p[]$ (упорядочим их, т.е. пусть $k_1 < k_2$). Тогда ответом на данный запрос будет минимум в массиве $\rm lcp$, взятый на отрезке $[k_1; k_2-1]$. В самом деле, переход от суффикса $i$ к суффиксу $j$ можно заменить целой цепочкой переходов, начинающейся с суффикса $i$ и заканчивающейся в суффиксе $j$, но включающей в себя все промежуточные суффиксы, находящиеся в порядке сортировки между ними.

Таким образом, если мы имеем такой массив $\rm lcp$, то ответ на любой запрос наибольшего общего префикса сводится к запросу \bf{минимума на отрезке} массива $\rm lcp$. Эта классическая задача минимума на отрезке (range minimum query, RMQ) имеет множество решений с различными асимптотиками, описанные \algohref=rmq{здесь}.

Итак, основная наша задача --- \bf{построение} этого массива $\rm lcp$. Строить его мы будем по ходу алгоритма построения суффиксного массива: на каждой текущей итерации будем строить массив $\rm lcp$ для циклических подстрок текущей длины.

После нулевой итерации массив $\rm lcp$, очевидно, должен быть нулевым.

Пусть теперь мы выполнили $k-1$-ю итерацию, получили от неё массив $\rm lcp^\prime$, и должны на текущей $k$-й итерации пересчитать этот массив, получив новое его значение $\rm lcp$. Как мы помним, в алгоритме построения суффиксного массива циклические подстроки длины $2^k$ разбивались пополам на две подстроки длины $2^{k-1}$; воспользуемся этим же приёмом и для построения массива $\rm lcp$.

Итак, пусть на текущей итерации алгоритм вычисления суффиксного массива выполнил свою работу, нашёл новое значение перестановки $p[]$ подстрок. Будем теперь идти по этому массиву и смотреть пары соседних подстрок: $p[i]$ и $p[i+1]$, $i=0 \ldots n-2$. Разбивая каждую подстроку пополам, мы получаем две различных ситуации: 1) первые половинки подстрок в позициях $p[i]$ и $p[i+1]$ различаются, и 2) первые половинки совпадают (напомним, такое сравнение можно легко производить, просто сравнивая номера классов $c[]$ с предыдущей итерации). Рассмотрим каждый из этих случаев отдельно.

1) Первые половинки подстрок различались. Заметим, что тогда на предыдущем шаге эти первые половинки необходимо были соседними. В самом деле, классы эквивалентности не могли исчезать (а могут только появляться), поэтому все различные подстроки длины $2^{k-1}$ дадут (в качестве первых половинок) на текущей итерации различные подстроки длины $2^k$, и в том же порядке. Таким образом, для определения ${\rm lcp}[i]$ в этом случае надо просто взять соответствующее значение из массива $\rm lcp^\prime$.

2) Первые половинки совпадали. Тогда вторые половинки могли как совпадать, так и различаться; при этом, если они различаются, то они совсем не обязательно должны были быть соседними на предыдущей итерации. Поэтому в этом случае нет простого способа определить ${\rm lcp}[i]$. Для его определения надо поступить так же, как мы и собираемся потом вычислять наибольший общий префикс для любых двух суффиксов: надо выполнить запрос минимума (RMQ) на соответствующем отрезке массива $\rm lcp^\prime$.

Оценим \bf{асимптотику} такого алгоритма. Как мы видели при разборе этих двух случаев, только второй случай даёт увеличение числа классов эквивалентности. Иными словами, можно говорить о том, что каждый новый класс эквивалентности появляется вместе с одним запросом RMQ. Поскольку всего классов эквивалентности может быть до $n$, то и искать минимум мы должны за асимптотику $O(\log n)$. А для этого надо использовать уже какую-то структуру данных для минимума на отрезке; эту структуру данных надо будет строить заново на каждой итерации (которых всего $O(\log n)$). Хорошим вариантом структуры данных будет \bf{\algohref=segment_tree{Дерево отрезков}}: его можно построить за $O(n)$, а потом выполнять запросы за $O(\log n)$, что как раз и даёт нам итоговую асимптотику $O(n \log n)$.

\bf{Реализация:}

\code
int lcp[maxlen], lcpn[maxlen], lpos[maxlen], rpos[maxlen];
memset (lcp, 0, sizeof lcp);
for (int h=0; (1<<h)<n; ++h) {
	for (int i=0; i<n; ++i)
		rpos[c[p[i]]] = i;
	for (int i=n-1; i>=0; --i)
		lpos[c[p[i]]] = i;

	... все действия по построению суфф. массива, кроме последней строки (memcpy) ...

	rmq_build (lcp, n-1);
	for (int i=0; i<n-1; ++i) {
		int a = p[i],  b = p[i+1];
		if (c[a] != c[b])
			lcpn[i] = lcp[rpos[c[a]]];
		else {
			int aa = (a + (1<<h)) % n,  bb = (b + (1<<h)) % n;
			lcpn[i] = (1<<h) + rmq (lpos[c[aa]], rpos[c[bb]]-1);
			lcpn[i] = min (n, lcpn[i]);
		}
	}
	memcpy (lcp, lcpn, (n-1) * sizeof(int));

	memcpy (c, cn, n * sizeof(int));
}
\endcode

Здесь помимо массива $\rm lcp[]$ вводится временный массив $\rm lcpn[]$ с его новым значением. Также поддерживается массив $\rm pos[]$, который для каждой подстроки хранит её позицию в перестановке $p[]$. Функция $\rm rmq\_build$ --- некоторая функция, строящая структуру данных для минимума по массиву-первому аргументу, размер его передаётся вторым аргументом. Функция $\rm rmq$ возвращает минимум на отрезке: с первого аргумента по второй включительно.

Из самого алгоритма построения суффиксного массива пришлось только вынести копирование массива $c[]$, поскольку во время вычисления $\rm lcp$ нам понадобятся старые значения этого массива.

Стоит отметить, что наша реализация находит длину общего префикса для \bf{циклических подстрок}, в то время как на практике чаще бывает нужной длина общего префикса для суффиксов в их обычном понимании. В этом случае надо просто ограничить значения $\rm lcp$ по окончании работы алгоритма:

\code
for (int i=0; i<n-1; ++i)
	lcp[i] = min (lcp[i], min (n-p[i], n-p[i+1]));
\endcode

Для \bf{любых} двух суффиксов длину их наибольшего общего префикса теперь можно найти как минимум на соответствующем отрезке массива $\rm lcp$:

\code
for (int i=0; i<n; ++i)
	pos[p[i]] = i;
rmq_build (lcp, n-1);

... поступил запрос (i,j) на нахождение LCP ...
int result = rmq (min(i,j), max(i,j)-1);
\endcode


\h3{ Количество различных подстрок }

Выполним \bf{препроцессинг}, описанный в предыдущем разделе: за $O(n \log n)$ времени и $O(n)$ памяти мы для каждой пары соседних в порядке сортировки суффиксов найдём длину их наибольшего общего префикса. Найдём теперь по этой информации количество различных подстрок в строке.

Для этого будем рассматривать, какие новые подстроки начинаются в позиции $p[0]$, затем в позиции $p[1]$, и т.д. Фактически, мы берём очередной в порядке сортировки суффикс и смотрим, какие его префиксы дают новые подстроки. Тем самым мы, очевидно, не упустим из виду никакие из подстрок.

Пользуясь тем, что суффиксы у нас уже отсортированы, нетрудно понять, что текущий суффикс $p[i]$ даст в качестве новых подстрок все свои префиксы, кроме совпадающих с префиксами суффикса $p[i-1]$. Т.е. все его префиксы, кроме ${\rm lcp}[i-1]$ первых, дадут новые подстроки. Поскольку длина текущего суффикса равна $n-p[i]$, то окончательно получаем, что текущий суффикс $p[i]$ даёт $n-p[i]-{\rm lcp}[i-1]$ новых подстрок. Суммируя это по всем суффиксам (для самого первого, $p[0]$, отнимать нечего --- прибавится просто $n-p[0]$), получаем \bf{ответ} на задачу:

$$ \sum_{i=0}^n (n - p[i]) - \sum_{i=0}^{n-1} {\rm lcp}[i] $$



\h2{ Задачи в online judges }

Задачи, которые можно решить, используя суффиксный массив:

\ul{

\li \href=http://uva.onlinejudge.org/index.php?option=onlinejudge&page=show_problem&problem=1620{UVA #10679 \bf{"I Love Strings!!!"} ~~~~ [сложность: средняя]}

}