\h1{ Поиск в ширину }

Поиск в ширину (обход в ширину, breadth-first search) --- это один из основных алгоритмов на графах.

В результате поиска в ширину находится путь кратчайшей длины в невзвешенном графе, т.е. путь, содержащий наименьшее число рёбер.

Алгоритм работает за $O (n+m)$, где $n$ --- число вершин, $m$ --- число рёбер.


\h2{ Описание алгоритма }

На вход алгоритма подаётся заданный граф (невзвешенный), и номер стартовой вершины $s$. Граф может быть как ориентированным, так и неориентированным, для алгоритма это не важно.

Сам алгоритм можно понимать как процесс "поджигания" графа: на нулевом шаге поджигаем только вершину $s$. На каждом следующем шаге огонь с каждой уже горящей вершины перекидывается на всех её соседей; т.е. за одну итерацию алгоритма происходит расширение "кольца огня" в ширину на единицу (отсюда и название алгоритма).

Более строго это можно представить следующим образом. Создадим очередь $q$, в которую будут помещаться горящие вершины, а также заведём булевский массив $\rm used[]$, в котором для каждой вершины будем отмечать, горит она уже или нет (или иными словами, была ли она посещена).

Изначально в очередь помещается только вершина $s$, и $\rm used[s] = true$, а для всех остальных вершин $\rm used[] = false$. Затем алгоритм представляет собой цикл: пока очередь не пуста, достать из её головы одну вершину, просмотреть все рёбра, исходящие из этой вершины, и если какие-то из просмотренных вершин ещё не горят, то поджечь их и поместить в конец очереди.

В итоге, когда очередь опустеет, обход в ширину обойдёт все достижимые из $s$ вершины, причём до каждой дойдёт кратчайшим путём. Также можно посчитать длины кратчайших путей (для чего просто надо завести массив длин путей $d[]$), и компактно сохранить информацию, достаточную для восстановления всех этих кратчайших путей (для этого надо завести массив "предков" $p[]$, в котором для каждой вершины хранить номер вершины, по которой мы попали в эту вершину).


\h2{ Реализация }

Реализуем вышеописанный алгоритм на языке C++.

Входные данные:

\code

vector < vector<int> > g; // граф
int n; // число вершин
int s; // стартовая вершина (вершины везде нумеруются с нуля)

// чтение графа
...
\endcode

Сам обход:

\code

queue<int> q;
q.push (s);
vector<bool> used (n);
vector<int> d (n), p (n);
used[s] = true;
p[s] = -1;
while (!q.empty()) {
	int v = q.front();
	q.pop();
	for (size_t i=0; i<g[v].size(); ++i) {
		int to = g[v][i];
		if (!used[to]) {
			used[to] = true;
			q.push (to);
			d[to] = d[v] + 1;
			p[to] = v;
		}
	}
}
\endcode

Если теперь надо восстановить и вывести кратчайший путь до какой-то вершины $\rm to$, это можно сделать следующим образом:

\code

if (!used[to])
	cout << "No path!";
else {
	vector<int> path;
	for (int v=to; v!=-1; v=p[v])
		path.push_back (v);
	reverse (path.begin(), path.end());
	cout << "Path: ";
	for (size_t i=0; i<path.size(); ++i)
		cout << path[i] + 1 << " ";
}
\endcode


\h2{ Приложения алгоритма }

\ul{

\li Поиск \bf{кратчайшего пути} в невзвешенном графе.

\li Поиск \bf{компонент связности} в графе за $O(n+m)$.

Для этого мы просто запускаем обход в ширину от каждой вершины, за исключением вершин, оставшихся посещёнными ($\rm used=true$) после предыдущих запусков. Таким образом, мы выполняем обычный запуск в ширину от каждой вершины, но не обнуляем каждый раз массив $\rm used[]$, за счёт чего мы каждый раз будем обходить новую компоненту связности, а суммарное время работы алгоритма составит по-прежнему $O(n+m)$ (такие несколько запусков обхода на графе без обнуления массива $\rm used$ называются серией обходов в ширину).

\li Нахождения решения какой-либо задачи (игры) \bf{с наименьшим числом ходов}, если каждое состояние системы можно представить вершиной графа, а переходы из одного состояния в другое --- рёбрами графа.

Классический пример --- игра, где робот двигается по полю, при этом он может передвигать ящики, находящиеся на этом же поле, и требуется за наименьшее число ходов передвинуть ящики в требуемые позиции. Решается это обходом в ширину по графу, где состоянием (вершиной) является набор координат: координаты робота, и координаты всех коробок.

\li Нахождение кратчайшего пути в \bf{0-1-графе} (т.е. графе взвешенном, но с весами равными только 0 либо 1): достаточно немного модифицировать поиск в ширину: если текущее ребро нулевого веса, и происходит улучшение расстояния до какой-то вершины, то эту вершину добавляем не в конец, а в начало очереди.

\li Нахождение \bf{кратчайшего цикла} в ориентированном невзвешенном графе: производим поиск в ширину из каждой вершины; как только в процессе обхода мы пытаемся пойти из текущей вершины по какому-то ребру в уже посещённую вершину, то это означает, что мы нашли кратчайший цикл, и останавливаем обход в ширину; среди всех таких найденных циклов (по одному от каждого запуска обхода) выбираем кратчайший.

\li Найти все рёбра, лежащие \bf{на каком-либо кратчайшем пути} между заданной парой вершин $(a,b)$. Для этого надо запустить 2 поиска в ширину: из $a$, и из $b$. Обозначим через $d_a[]$ массив кратчайших расстояний, полученный в результате первого обхода, а через $d_b[]$ --- в результате второго обхода. Теперь для любого ребра $(u,v)$ легко проверить, лежит ли он на каком-либо кратчайшем пути: критерием будет условие $d_a[u] + 1 + d_b[v] = d_a[b]$.

\li Найти все вершины, лежащие \bf{на каком-либо кратчайшем пути} между заданной парой вершин $(a,b)$. Для этого надо запустить 2 поиска в ширину: из $a$, и из $b$. Обозначим через $d_a[]$ массив кратчайших расстояний, полученный в результате первого обхода, а через $d_b[]$ --- в результате второго обхода. Теперь для любой вершины $v$ легко проверить, лежит ли он на каком-либо кратчайшем пути: критерием будет условие $d_a[v] + d_b[v] = d_a[b]$.

\li Найти \bf{кратчайший чётный путь} в графе (т.е. путь чётной длины). Для этого надо построить вспомогательный граф, вершинами которого будут состояния $(v,c)$, где $v$ --- номер текущей вершины, $c = 0 \ldots 1$ --- текущая чётность. Любое ребро $(a,b)$ исходного графа в этом новом графе превратится в два ребра $((u,0),(v,1))$ и $((u,1),(v,0))$. После этого на этом графе надо обходом в ширину найти кратчайший путь из стартовой вершины в конечную, с чётностью, равной 0.

}



\h2{ Задачи в online judges }

Список задач, которые можно сдать, используя обход в ширину:

TODO