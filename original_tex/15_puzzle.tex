\h1{Игра Пятнашки: существование решения}

Напомним, что игра представляет собой поле $4$ на $4$, на котором расположены $15$ фишек, пронумерованных числами от $1$ до $15$, а одно поле оставлено пустым. Требуется, передвигая на каждом шаге какую-либо фишку на свободную позицию, прийти в конце концов к следующей позиции:
$$ \matrix{
1&2&3&4\cr
5&6&7&8\cr
9&10&11&12\cr
13&14&15&\bigcirc\cr
} $$

Игру Пятнашки ("15 puzzle") изобрёл в 1880 г. Нойес Чэпман (Noyes Chapman).

\h2{Существование решения}

Здесь мы рассмотрим такую задачу: по данной позиции на доске сказать, существует ли последовательность ходов, приводящая к решению, или нет.

Пусть дана некоторая позиция на доске:

$$ \matrix{
a_1 & a_2 & a_3 & a_4 \cr 
a_5 & a_6 & a_7 & a_8 \cr 
a_9 & a_{10} & a_{11} & a_{12} \cr 
a_{13} & a_{14} & a_{15} & a_{16} \cr 
} $$
где один из элементов равен нулю и обозначает пустую клетку $a_z = 0$.

Рассмотрим перестановку:
$$ a_1 a_2 \ldots a_{z-1} a_{z+1} \ldots a_{15} a_{16} $$
(т.е. перестановка чисел, соответствующая позиции на доске, без нулевого элемента)

Обозначим через $N$ количество инверсий в этой перестановке (т.е. количество таких элементов $a_i$ и $a_j$, что $i < j$, но $a_i > a_j$).

Далее, пусть $K$ --- номер строки, в которой находится пустой элемент (т.е. в наших обозначениях $K = (z-1)\ {\rm div}\ 4 + 1)$.

Тогда, \bf{решение существует тогда и только тогда, когда $N+K$ чётно}.

\h2{Реализация}

Проиллюстрируем указанный выше алгоритм с помощью программного кода:
\code
int a[16];
for (int i=0; i<16; ++i)
	cin >> a[i];

int inv = 0;
for (int i=0; i<16; ++i)
	if (a[i])
		for (int j=0; j<i; ++j)
			if (a[j] > a[i])
				++inv;
for (int i=0; i<16; ++i)
	if (a[i] == 0)
		inv += 1 + i / 4;

puts ((inv & 1) ? "No Solution" : "Solution Exists");
\endcode

\h2{Доказательство}
Джонсон (Johnson) в 1879 г. доказал, что если $N+K$ нечётно, то решения не существует, а Стори (Story) в том же году доказал, что все позиции, для которых $N+K$ чётно, имеют решение.

Однако оба эти доказательства были достаточно сложны.

В 1999 г. Арчер (Archer) предложил значительно более простое доказательство (скачать его статью можно \href=http://www.cs.cmu.edu/afs/cs/academic/class/15859-f01/www/notes/15-puzzle.pdf{здесь}).
