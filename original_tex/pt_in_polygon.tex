<h1>Проверка точки на принадлежность выпуклому многоугольнику</h1>

<p>Дан выпуклый многоугольник с N вершинами, координаты всех вершин целочисленны (хотя это не меняет суть решения); вершины заданы в порядке обхода против часовой стрелки (в противном случае нужно просто отсортировать их). Поступают запросы - точки, и требуется для каждой точки определить, лежит она внутри этого многоугольника или нет (границы многоугольника включаются). На каждый запрос будем отвечать в режиме on-line за O (log N). Предварительная обработка многоугольника будет выполняться за O (N).</p>
<h2>Алгоритм</h2>
<p>Решать будем <b>бинарным поиском по углу</b>.</p>
<p>Один из вариантов решения таков. Выберем точку с наименьшей координатой X (если таких несколько, то выбираем самую нижнюю, т.е. с наименьшим Y). Относительно этой точки, обозначим её Zero, все остальные вершины многоугольника лежат в правой полуплоскости. Далее, заметим, что все вершины многоугольника уже упорядочены по углу относительно точки Zero (это вытекает из того, что многоугольник выпуклый, и уже упорядочен против часовой стрелки), причём все углы находятся в промежутке (-&pi;/2 ; &pi;/2].</p>
<p>Пусть поступает очередной запрос - некоторая точка P. Рассмотрим её полярный угол относительно точки Zero. Найдём бинарным поиском две такие соседние вершины L и R многоугольника, что полярный угол P лежит между полярными углами L и R. Тем самым мы нашли тот сектор многоугольника, в котором лежит точка P, и нам остаётся только проверить, лежит ли точка P в треугольнике (Zero,L,R). Это можно сделать, например, с помощью <algohref=oriented_area>Ориентированной площади треугольника и Предиката "По часовой стрелке"</algohref>, достаточно посмотреть, по часовой стрелке или против находится тройка вершин (R,L,P).</p>
<p>Таким образом, мы за O (log N) находим сектор многоугольника, а затем за O (1) проверяем принадлежность точки треугольнику, и, следовательно, требуемая асимптотика достигнута. Предварительная обработка многоугольника заключается только в том, чтобы предпосчитать полярные углы для всех точек, хотя, эти вычисления тоже можно перенести на этап бинарного поиска.</p>
<h2>Замечания по реализации</h2>
<p>Чтобы определять полярный угол, можно воспользоваться стандартной функцией atan2. Тем самым мы получим очень короткое и простое решение, однако взамен могут возникнуть проблемы с точностью.</p>
<p>Учитывая, что изначально все координаты являются целочисленными, можно получить решение, вообще не использующее дробной арифметики.</p>
<p>Заметим, что полярный угол точки (X,Y) относительно начала координат однозначно определяется дробью Y/X, при условии, что точка находится в правой полуплоскости. Более того, если у одной точки полярный угол меньше, чем у другой, то и дробь Y1/X1 будет меньше Y2/X2, и обратно.</p>
<p>Таким образом, для сравнения полярных углов двух точек нам достаточно сравнить дроби Y1/X1 и Y2/X2, что уже можно выполнить в целочисленной арифметике.</p>
<h2>Реализация</h2>
<p>Эта реализация предполагает, что в данном многоугольнике нет повторяющихся вершин, и площадь многоугольника ненулевая.</p>
<code>struct pt {
	int x, y;
};

struct ang {
	int a, b;
};

bool operator &lt; (const ang & p, const ang & q) {
	if (p.b == 0 && q.b == 0)
		return p.a &lt; q.a;
	return p.a * 1ll * q.b &lt; p.b * 1ll * q.a;
}

long long sq (pt & a, pt & b, pt & c) {
	return a.x*1ll*(b.y-c.y) + b.x*1ll*(c.y-a.y) + c.x*1ll*(a.y-b.y);
}

int main() {

	int n;
	cin >> n;
	vector&lt;pt> p (n);
	int zero_id = 0;
	for (int i=0; i&lt;n; ++i) {
		scanf ("%d%d", &p[i].x, &p[i].y);
		if (p[i].x &lt; p[zero_id].x || p[i].x == p[zero_id].x && p[i].y &lt; p[zero_id].y)
			zero_id = i;
	}
	pt zero = p[zero_id];
	rotate (p.begin(), p.begin()+zero_id, p.end());
	p.erase (p.begin());
	--n;

	vector&lt;ang> a (n);
	for (int i=0; i&lt;n; ++i) {
		a[i].a = p[i].y - zero.y;
		a[i].b = p[i].x - zero.x;
		if (a[i].a == 0)
			a[i].b = a[i].b &lt; 0 ? -1 : 1;
	}

	for (;;) {
		pt q; // очередной запрос
		bool in = false;
		if (q.x >= zero.x)
			if (q.x == zero.x && q.y == zero.y)
				in = true;
			else {
				ang my = { q.y-zero.y, q.x-zero.x };
				if (my.a == 0)
					my.b = my.b &lt; 0 ? -1 : 1;
				vector&lt;ang>::iterator it = upper_bound (a.begin(), a.end(), my);
				if (it == a.end() && my.a == a[n-1].a && my.b == a[n-1].b)
					it = a.end()-1;
				if (it != a.end() && it != a.begin()) {
					int p1 = int (it - a.begin());
					if (sq (p[p1], p[p1-1], q) &lt;= 0)
						in = true;
				}
			}
		puts (in ? "INSIDE" : "OUTSIDE");
	}

}</code>